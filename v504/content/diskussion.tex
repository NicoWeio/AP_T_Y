\section{Diskussion}
\label{sec:Diskussion}
Eine zu beachtende errechnete Größe ist die Austrittsarbeit. Denn diese weicht vom Literaturwert\cite{lit}
\begin{equation*}
    e_0\phi_\text{Lit} =  \SI{4.54}{\electronvolt} - \SI{4.6}{\electronvolt}
\end{equation*}
stark ab. Das Verhältnis des errechneten Wertes und des Literaturwerts liegt bei
\begin{equation*}
    \eta = \frac{\SI{1.69}{\electronvolt}}{\SI{4.57}{\electronvolt}} \approx \num{0.3698} \; \text{,}
\end{equation*}
was einer Abweichung von ungefähr $\SI{63}{\percent}$ entspricht.
Da die beiden einzigen Variablen bei der Berechnung der Austrittsarbeit die Temperatur und der Sättigungsstrom sind, muss der Fehler an mindestens 
einer dieser beiden Größen liegen. Möglicherweise könnte es an einer ungenauen Bestimmung des Sättigungsstromes liegen, da dieser nur graphisch bestimmt worden ist. 
Außerdem könnte der Wert für den Strom nicht sauber an der Apperatur abgelesen worden sein.
Des Weiteren sind die beiden stark abweichenden Punkte in der Abbildung \ref{fig:exponent} auffällig. 
Diese beiden Ausreißer enstanden durch die Korrektur des Spannungsabfalls über das Nanoamperemeter.
Eine mögliche Fehlerquelle könnte eine falsche Skalierung des Nanoamperemeters sein. 
Diese fließen wegen der Korrektur \eqref{eqn:Korrektur} stark in die Berechung ein.
Ein weiterer Grund könnte eine Störung bei dem Messvorgang an dem Nanoamperemeter sein, da dieses sehr empfindlich auf Bewegungen oder 
benachbarte Objekte reagierte.