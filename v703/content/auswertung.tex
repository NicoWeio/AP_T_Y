\section{Auswertung}
Jegliche Fehlerrechnung wurde mit der python-Bibliothek uncertainties \cite{uncertainties} absolviert.
Trotz dessen sind die Formeln für die Unsicherheiten in den jeweiligen Abschnitten angegeben.
Allgemeine Rechnungen wurden mit der python-Bibliothek numpy \cite{numpy} automatisiert. 
Die graphischen Untersützungen wurden mit Hilfe der python-Bibliothek matplotlib \cite{matplotlib} erstellt.
\label{sec:Auswertung}
\subsection{Geiger-Müller Charakteristik}
Zur Bestimmung des Plateau-Bereichs des Zählrohrs wurde ein Spannung mit Schritten von $\symup{\Delta} U = \SI{10}{\volt}$ angelegt.
Dazu wurde die Teilchenanzahl pro Zeitintervall $N$ gemessen. 
Das Zeitintervall beträgt hierbei $\symup{\Delta} t = \SI{60}{\second}$.
Zur Berechnnung der Plateau-Ausgleichsgerade wurde das Intervall $\SI{370}{\volt}$ bis $\SI{630}{\volt}$ gewählt.
Somit besitzt die Charakteristik eine Plateau-Länge von 
\begin{equation*}
  U_\text{Plateau} = 260 V \;\text{.}
\end{equation*}
Mittels Rechnungen in python lassen sich die Parameter der Ausgleichsgerade 
\begin{equation}
  N = aU + b
\end{equation}
zu 
\begin{align*}
  a &= \SI{1.1378(2407)}{\per\minute\per\volt} = ( \num{1.1378(2407)}) * 10^4  \,\frac{\si{\percent}}{100 \, \si{\minute\volt}} \\
  b &= \SI{9590.7346(1218237)}{\per\minute}
\end{align*}
bestimmen.
\begin{figure}
  \centering
  \caption{Charakteristik des Halogenzählrohrs}
  \label{fig:char}
  \includegraphics{build/char.pdf}
\end{figure}
Die Unsicherheit lässt sich mit Hilfe der Beziehung 
\begin{equation}
  \symup{\Delta} N = \sqrt{N}
\end{equation}
ermitteln.
\subsection{Bestimmung der Totzeit}
In der Tabelle \ref{tab:Zaehlrate} sind die gemessen Impulse pro $120$ Sekunden $N$ aufgetragen, wobei sich $N_1$ auf die gemessenen Impulse der ersten Quelle und $N_2$
auf die gemessenen Impulse der zweiten Quelle bezieht. $N_\text{1+2}$ beschreibt die Impulse, welche von beiden Quellen gleichzeitig ausgestrahlt werden.
\begin{table}
  \centering
  \caption{Gemessene Impulse der Konfigurationen}
  \label{tab:Zaehlrate}
  \begin{tabular}{S[table-format = 5] S[table-format = 5] S[table-format = 6]}
    \toprule
    {$N_1 \mathbin{/} (120 \si{\second})^{-1}$} & {$N_2 \mathbin{/} (120 \si{\second})^{-1} $} & {$N_\text{1+2} \mathbin{/} (120 \si{\second})^{-1}$} \\
    \midrule
    96041   & 76518   & 158479 \\
    \bottomrule
  \end{tabular}
\end{table}
Mittels Gleichung \eqref{eqn:totzeit} lässt sich die Totzeit $T$ zu 
\begin{equation*}
  T \approx \SI{110(50)}{\micro\second}     
\end{equation*}
bestimmen.
Die Unsicherheit der Totzeit wird mit Hilfe der Gaußschen Fehlerfortpflanzung berechnet 
\begin{equation}
  \symup{\Delta} T = \sqrt{\left(1+ \frac{2n_2-2n_\text{1+2}}{4n_1^2 n_2}\right)^2 n_1 + 
  \left(1+ \frac{2n_1-2n_\text{1+2}}{4n_1 n_2^2}\right)^2 n_2 + 
  \frac{n_\text{1+2}}{n_1^2 n_2^2}  }
\end{equation}
An dem Oszilloskop kann eine Totzeit von 
\begin{equation*}
  T \approx \SI{160}{\micro\second}
\end{equation*}
abgelesen werden.
\subsection{Bestimmung des Zählerrohrstroms}
Für die freigesetzten Ladungen pro einfallendes Teilchen $Z$ in Abhängigkeit des Zählrohrstroms $I$ ergibt sich die Tabelle \ref{tab:eifausf}.
Ebenfalls sind die freigesetzten Ladungen pro einfallendes Teilchen in Abbildung \ref{fig:free} gegen den Zählerrohrstrom aufgetragen.
 \begin{table}
  \centering
  \caption{freigesetzte Ladungen pro einfallendes Teilchen}
  \label{tab:eifausf}
  \begin{tabular}
    {S[table-format = 1.1] @{${}\pm{}$} S[table-format = 1.2]
     S[table-format = 2.4] @{${}\pm{}$} S[table-format = 1.4]
    }
    \toprule
    \multicolumn{2}{c}{$I \mathbin{/} \si{\micro\ampere}$}       &
    \multicolumn{2}{c}{$Z$ in Mrd.}                              \\
    \midrule
    0.3 & 0.05 & 11.4209 & 1.9035  \\
    0.4 & 0.05 & 14.9871 & 1.8734  \\
    0.7 & 0.05 & 25.5401 & 1.8243  \\
    0.8 & 0.05 & 29.5136 & 1.8456  \\
    1.0 & 0.05 & 36.7724 & 1.8386  \\
    1.3 & 0.05 & 47.4824 & 1.8262  \\
    1.4 & 0.05 & 49.9654 & 1.7845  \\
    1.8 & 0.05 & 58.3773 & 1.6216  \\
    \bottomrule
    \end{tabular}
\end{table} 
\begin{figure}
  \centering
  \caption{Freigesetze Teilchen pro eingefallende Teilchen}
  \label{fig:free}
  \includegraphics{build/cur.pdf}
\end{figure}