\section{Diskussion}
\label{sec:Diskussion}
Zunächst wird auffällig, dass das Verhältnis aus der mit dem Oszilloskop bestimmten und der errechneten Totzeit klein ist, da
das Verhältnis
\begin{equation*}
    \frac{T_\text{berechnet}}{T_\text{Oszilloskop}} = \SI{68.75}{\percent} \; \text{.}
\end{equation*}
beträgt.
Im Hinblick auf die Problematik könnte die Ursache an dem Oszilloskop liegen. 
Die dortige Visualisierung zeugt von geringer Genauigkeit, denn die einzelnen Kurven sind nicht komplett gebündelt, sondern weichen von einander ab.
Somit liegt die Vermutung nahe, dass die optimalen Ablesepunkte zur Bestimmung der Totzeit woanders liegen und somit eine relativ große Abweichung zu Stande kam.
Andererseits könnte die starke Abweichung an der Tatsache liegen, dass eine Annäherung zur Berechnung der Totzeit verwendet wurde.\\
Neben der Totzeit wird die Plateau-Steigung negativ auffällig, da diese  größer als eins ist ($\approx \SI{1.13}{\per\minute\per\volt}$).
Jedoch ist die Plateau-Länge nur sehr schwierig zu bestimmen, da der Graph in Abbildung \ref{fig:char}
schweren Schwankungen unterliegt. 
Somit ist der Beginn des Plateaus nicht ganz eindeutig, wodurch eine optimale Wahl der Plateau-Länge schwierig zu bestimmen ist.