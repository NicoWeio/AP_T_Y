\section{Auswertung}
\label{sec:Auswertung}
Jegliche Fehlerrechnung wurde mit der python-Bibliothek uncertainties \cite{uncertainties} absolviert.
Allgemeine Rechnungen wurden mit der python-Bibliothek numpy \cite{numpy} automatisiert. 
Die graphischen Unterstützungen wurden mit Hilfe der python-Bibliothek matplotlib \cite{matplotlib} erstellt.\\
\subsection{Überprüfung des Reflexionsgesetzes}
Die Gleichung $REFERENZ$ besagt, dass der Ausfallswinkel immer dem Einfallswinkel gleicht.
Zur Überprüfung dieses Gesetzes wurden die gemessenen Einfallswinkel $\alpha_1$ und die Ausfallswinkel $\alpha_2$
, welche in der Tabelle \ref{tab:reflection} stehen, aufgenommen.
\begin{table}
    \centering
    \caption{Gemessene Einfallswinkel $\alpha_1$ und Ausfallswinkel $\alpha_2$}
    \label{tab:reflection}
    \begin{tabular} {S[table-format=2.0] S[table-format=2.1] }
        \toprule
        {$\alpha_1 \mathbin{/} \si{\degree}$} & {$\alpha_2 \mathbin{/} \si{\degree}$}\\
    \midrule
    20 & 20     \\
    30 & 29.5   \\
    40 & 39     \\
    50 & 49     \\
    60 & 59     \\
    70 & 69     \\
    80 & 79.5   \\
    \bottomrule
\end{tabular}
\end{table}
\subsection{Berechnung des Brechungsindex und die Lichtgeschwindigkeit in Plexiglas} \label{sec:refraction}
In der Tabelle \ref{tab:refraction} sind die Einfallswinkel $\alpha$ und die dazugehörigen 
Brechungswinkel $\beta$, welche in der planaren Plexiglasplatte enstanden sind, und der daraus berechnete Brechungsindex $n$ aufgelistet.
Der Brechungsindex lässt sich gemäß der Beziehung $REFERENZ$ errechnen.
\begin{table}
    \centering
    \caption{Gemessene Einfallswinkel $\alpha$ und Brechungsinkel $\beta$}
    \label{tab:refraction}
    \begin{tabular} {S[table-format=2.2] S[table-format=2.1] S[table-format=1.4] }
        \toprule
        {$\alpha \mathbin{/} \si{\degree}$} & {$\beta \mathbin{/} \si{\degree}$} & {$n_\text{Plexi}$}\\
    \midrule
    20 & 15   & 1.3215 \\
    30 & 18   & 1.6180 \\
    40 & 25.5 & 1.4931 \\
    50 & 31   & 1.4874 \\
    60 & 35   & 1.5099 \\
    70 & 37.5 & 1.5436 \\
    80 & 42   & 1.4718 \\
    \bottomrule
\end{tabular}
\end{table}
Aus den Einzelwerten ergibt sich ein Mittelwert von 
\begin{equation}
    \bar{n}_\text{Plexi} = \num{1.4921(118)} \label{eqn:refractonplexi} \; \text{.}
\end{equation}
Nach dem Brechungsgetz mit einem Brechungsindex von $\approx 1$ des optisch dünneren Medium, welches in unserem Fall Luft ist,
gilt für die Lichtgeschwindigkeit in dem optisch dichteren Medium 
\begin{equation}
    c_\text{Plexi} = \frac{c_\text{Luft}}{\bar{n}_\text{Plexi}} \; \text{.}
\end{equation}
Mit der Lichtgeschwindigkeit in Luft von $c_\text{Luft} = \SI{2.9979e8}{\metre\per\second}$ und dem eben berechnten 
Brechungsindex von Plexiglas \eqref{eqn:refractonplexi} lässt sich die Lichtgeschwindigkeit in Plexiglas zu 
\begin{equation*}
    c_\text{Plexi} = \SI{2.009(16)e8}{\metre\per\second}
\end{equation*}
berechnen.
\subsection{Bestimmung des Strahlenversatzes}
In der Tabelle \ref{tab:offset} sind erneut die gemessene Winkel aus dem Abschnitt \ref{sec:refraction} der Vollständigkeit halber aufgeführt.
Dazu sind die, mit der Gleichung $REFERENZ$ berechneten, Strahlenversätze $s$ aufgelistet.
Ebenso sind die Strahlenversätze $\tilde{s}$, welche mit Hilfe des Brechungsgetzes $REFERENZ$ und den Brechungswinkel $\tilde{\beta}$, 
welche mit Hilfe des Brechungsindex \eqref{eqn:refractonplexi} aus Abschnitt \ref{sec:refraction} brechnet worden sind, dort eingetragen.
\begin{table}
    \centering
    \caption{Gemessene Einfallswinkel $\alpha$ und Brechungsinkel $\beta$ und der daraus errechnete Strahlenversatz $s$ und $\tilde{s}$}
    \label{tab:offset}
    \begin{tabular} {S[table-format=2.0] S[table-format=2.1] S[table-format=2.2] S[table-format=1.4] S[table-format=1.4]}
        \toprule
        {$\alpha \mathbin{/} \si{\degree}$} & {$\beta \mathbin{/} \si{\degree}$} & {$\tilde{\beta} \mathbin{/} \si{\degree}$} &
        {$s \mathbin{/} \si{\metre}$} &  {$\tilde{s} \mathbin{/} \si{\metre}$}\\
    \midrule
    20 & 15   & 13.25 & 0.0053 & 0.0071 \\
    30 & 18   & 19.58 & 0.0128 & 0.0112 \\
    40 & 25.5 & 25.52 & 0.0162 & 0.0162 \\
    50 & 31   & 30.89 & 0.0222 & 0.0223 \\
    60 & 35   & 35.48 & 0.0302 & 0.0298 \\
    70 & 37.5 & 39.03 & 0.0396 & 0.0388 \\
    80 & 42   & 41.30 & 0.0485 & 0.0487 \\
    \bottomrule
\end{tabular}
\end{table}
\subsection{Bestimmung der Ablenkung}
Mit Hilfe der Gleichung $REFERENZ$ kann die Ablenkung $\delta$ bestimmt werden.
Hierzu wird der Brechungsindex von Kronglas benötigt, welcher den Wert $n_\text{Kron} = 1.555$\cite{Lit} hat.
Der benötigte Winkel $\beta_1$ kann mittels des Brechungsgetzes zu
\begin{equation*}
    \beta_1 = \arcsin \left( \frac{\sin \alpha_1 }{n} \right)
\end{equation*}
berechnet werden, wonach der Winkel $\beta_2$ durch 
\begin{equation*}
    \beta_2 = \gamma - \beta_1
\end{equation*}
bestimmt werden kann. Dabei gilt $\gamma = \ang{60;;}$.
In der Tablle \ref{tab:prisma} ist der gemessene Einfallswinkel $\alpha_1$, gemessene Ausfallswinkel des Lichtstrahls $\alpha_2$ 
und die Ablenkung $\delta$ aufgeführt, wobei die Indizes g und r für grün bzw. rot steht. 
\begin{table}
    \centering
    \caption{Gemessene Einfallswinkel $\alpha_1$ und Austrittswinkel $\alpha_2$ und die daraus errechnete Ablenkung $\delta$ für 
    grünes und rotes licht.}
    \label{tab:prisma}
    \begin{tabular} {S[table-format=2.0] S[table-format=2.1] S[table-format=2.2] S[table-format=1.4] S[table-format=1.4]}
        \toprule
        {$\alpha_1 \mathbin{/} \si{\degree}$} & {$\alpha_{2, \text{g}} \mathbin{/} \si{\degree}$} & {$\alpha_{2, \text{r}} \mathbin{/} \si{\degree}$} &
        {$\delta_\text{g} \mathbin{/} \si{\degree}$} &  {$\delta_\text{r} \mathbin{/} \si{\degree}$}\\
    \midrule
    27 & 90   & 86   & 57.0 & 53.0 \\
    35 & 66.5 & 65.5 & 41.5 & 40.5 \\
    43 & 55   & 54   & 38.0 & 37.0 \\
    51 & 46   & 45.5 & 37.0 & 36.5 \\
    59 & 33.5 & 33   & 32.5 & 32.0 \\
    67 & 33   & 32   & 40.0 & 39.0 \\
    \bottomrule
\end{tabular}
\end{table}
\subsection{Bestimmung der Wellenlänge mittels Beugung am Gitter}
Die gemessenen Beugungswinkel $\varphi$ verschiedener Beugungsordnungen $k$ bei verschiedenen Gitterkonstanten $d$ sind in 
der Tabelle \ref{tab:diffractionangle} aufgelistet.
Der Index g und r stehen dabei für grünes oder rotes Licht.
Aus den gemessenen Beungswinkeln und Beugungsordnungen lässt sich die Wellenlänge des verwendeten Lasers mittels $REFERENZ$ ermitteln.
\begin{table}
    \centering
    \caption{Gemessene Beugungswinkel verschiedener Beugungsordnungen von grünem und rotem Licht bei verschiedenen Gitterkonstanten.}
    \label{tab:diffractionangle}
    \begin{tabular} {S[table-format=1.0] 
                     S[table-format=2.1]  S[table-format=2.1]
                     S[table-format=2.1]  S[table-format=2.1] 
                     S[table-format=2.1]  S[table-format=2.1]}
        \toprule
        & 
        \multicolumn{2}{c}{$d = \num{1e-5}$} & 
        \multicolumn{2}{c}{$d = \sfrac{1}{3}\cdot\num{e-5}$} & 
        \multicolumn{2}{c}{$d = \sfrac{1}{6}\cdot\num{e-5}$}\\
        \cmidrule(lr){2-3}\cmidrule(lr){4-5}\cmidrule(lr){6-7}
        {$k$} 
        & {$\varphi_\text{g} \mathbin{/} \si{\degree}$} & {$\varphi_\text{r} \mathbin{/} \si{\degree}$}
        & {$\varphi_\text{g} \mathbin{/} \si{\degree}$} & {$\varphi_\text{r} \mathbin{/} \si{\degree}$}
        & {$\varphi_\text{g} \mathbin{/} \si{\degree}$} & {$\varphi_\text{r} \mathbin{/} \si{\degree}$} \\
    \midrule
    1 & 3    & 3.5  & 8.5  &  10.1 & 17.9 &  21.5 \\
    2 & 5.8  & 7    & 17.3 &  21   & {-}  &   {-} \\
    3 & 8.7  & 10.5 & 21.9 &  33   & {-}  &   {-} \\
    4 & 11.6 & 14   & {-}  &  {-}  & {-}  &   {-} \\
    5 & 14.7 & 17.7 & {-}  &  {-}  & {-}  &   {-} \\
    \bottomrule
\end{tabular}
\end{table}
Die aus den Messwerten ermittelten Wellenlänge sind in der Tabelle \ref{tab:diffractionwavelength} aufgeführt.
\begin{table}
    \centering
    \caption{Berechnete Wellenlängen zu den gemessenen zu den gemessenen Beugungswinkeln von grünem und rotem Licht.}
    \label{tab:diffractionwavelength}
    \begin{tabular} {S[table-format=1.0] 
                     S[table-format=3.2]  S[table-format=3.2]
                     S[table-format=3.2]  S[table-format=3.2] 
                     S[table-format=3.2]  S[table-format=3.2]}
        \toprule
        & 
        \multicolumn{2}{c}{$d = \num{1e-5}$} & 
        \multicolumn{2}{c}{$d = \sfrac{1}{3}\cdot\num{e-5}$} & 
        \multicolumn{2}{c}{$d = \sfrac{1}{6}\cdot\num{e-5}$}\\
        \cmidrule(lr){2-3}\cmidrule(lr){4-5}\cmidrule(lr){6-7}
        {$k$} 
        & {$\lambda_\text{g} \mathbin{/} \si{\nano\metre}$} & {$\lambda_\text{r} \mathbin{/} \si{\nano\metre}$}
        & {$\lambda_\text{g} \mathbin{/} \si{\nano\metre}$} & {$\lambda_\text{r} \mathbin{/} \si{\nano\metre}$}
        & {$\lambda_\text{g} \mathbin{/} \si{\nano\metre}$} & {$\lambda_\text{r} \mathbin{/} \si{\nano\metre}$} \\
    \midrule
    1 & 523.36 & 610.49 & 492.70 & 584.56 & 512.26 & 610.84\\
    2 & 505.28 & 609.35 & 495.62 & 597.28 & {-}    & {-}   \\
    3 & 504.20 & 607.45 & 414.43 & 605.15 & {-}    & {-}   \\
    4 & 502.69 & 604.80 & {-}    & {-}    & {-}    & {-}   \\
    5 & 507.52 & 608.07 & {-}    & {-}    & {-}    & {-}   \\
    \bottomrule
\end{tabular}
\end{table}
Über alle berechneten Wellenlängen des jeweiligen Lichts ergibt sich der Mittelwert der Wellenlänge von dem grünem bzw. rotem Laser zu
\begin{align*}
    \bar{\lambda}_\text{g} & = \SI{495.34(5504)}{\nano\metre} \\
    \bar{\lambda}_\text{r} & = \SI{604.22(6714)}{\nano\metre} \, \text{.}
\end{align*}