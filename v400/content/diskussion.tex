\section{Diskussion}
\label{sec:Diskussion}
Bei Betrachtung der Tabelle \ref{tab:reflection} fällt auf, dass die Ausfallswinkel bis auf den ersten gemessenen Winkel um $\ang{0.5;;}$ bis 
$\ang{1;;}$ abweichen. Möglichen Ursachen für die Abweichung könnte an einem beschädigten Spiegel liegen, welcher Unebenheiten auf dessen 
Oberfläche vorweisen könnte.\\
Der berechnete Brechungsindex von Plexiglas $n_\text{Plexi}=1.4921$ aus dem Abschnitt \ref{sec:refraction} weist von dem Literaturwert 
$n_\text{Plexi, Lit} = 1.49$\cite{lit} eine Abweichung von $\eta_n = \SI{0.14}{\percent}$ auf. 
Diese Abweichung könnte an der bereits abgenutzen planaren Plexiglas liegen, da sichtbare Fettflecken auf der Oberfläche erkenntlich wurden.
Jedoch ist die Abweichung mit etwa $\SI{4}{\percent}$ hinreichend klein, so dass gesagt werden kann, dass der Brechungsindex von Plexiglas mit einer hohen 
Präzision bestimmen werden konnte.\\ 
Die berechneten Strahlenversätze aus der Tabelle \ref{tab:offset} weisen eine hohe Übereinstimmung auf.
Dies liefert den Anschein, dass es eine beinahe identische Präzision liefert wenn der Winkel $\beta$ gemessen oder mit Hilfe des gegebenen
Brechungsindex berechnet wird.\\
Die experimentell bestimmte Wellenlänge des grünen Lasers $\lambda_\text{g} = \SI{495.34}{\nano\metre}$ weicht von der angegebenen 
Wellenlänge $\lambda_{\text{g, Lit}} = \SI{532}{\nano\metre}$ um  $\eta_{\lambda_\text{g}} = \SI{6.95}{\percent}$ ab; die  des roten Lasers 
$\lambda_\text{r} = \SI{604.22}{\nano\metre}$ weicht von dem angegebenen Wert $\lambda_{\text{r, Lit}} = \SI{635}{\nano\metre}$ um
$\eta_{\lambda_\text{r}} = \SI{4.85}{\percent}$ ab.
Erneut sind die Abweichungen sehr gering, wodurch gesagt werden kann, dass auch bei diesem Versuchsteil eine hohe Genauigkeit erzielt werden konnte.
Jedoch lässt sich bei Betrachtung der Wellenlängen in der Tabelle \ref{tab:diffractionwavelength} bebobachten, dass diese immer kleiner als der angegebenene Wert ist, 
so dass sich dort eine systematischer Fehler vermutet werden kann. \\
Im Allgemeinen kann gesagt werden, dass die Ungenauigkeiten durch die Messaperatur, welche händisch per Augenmaß ausgerichtet wird, Zustande kamen.
Dadurch kann schon eine kleine fehlerhafte Ausrichtung des Messskalen zu einer beeinflussenden Ungenauigkeit werden.