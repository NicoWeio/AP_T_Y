\section{Diskussion}
\label{sec:Diskussion}
In dem Abschnitt \ref{sec:filter} wurden die beiden Abbildungen \ref{fig:frequence} und \ref{fig:frequencefine} erstellt, da in der ersten Abbildungen keine 
Struktur der Filterkurve des Selektiv-Verstärkers ersichtlich wurde. Somit würden auch die interessanten Größen wie z.B. die Güte sehr ungenau bestimmt werden können.
Folglich wurde die zweite Abbildung erstellt, um die Filterkurve in dem interssanten Bereich erkennen zu können, wodurch auch die Bestimmung der Güte deutlich präziser wurde.
Mögliche starke systematische Fehler in der Messung könnten an dem digitalen Messgerät liegen, da dieses bei Umstellung von einer auf eine andere Skala Sprünge in der
Spannung vorwies. Somit sind die Sprünge in der Tabelle \ref{tab:frequence} und in der Abbildung \ref{fig:frequence} aller Messwerte erkenntlich.
Da in der Abbildung, welche den Bereich um die Resonanzfrequenz darstellt, keine Sprünge vorhanden ist, stärkt dies die These, dass die Fehler an dem Wechsel der
Skalierung an dem Messgerät liegen könnten, da dort nur mit einer Skala gemessen wurde. 
Die experimentell bestimmte Güte weicht von der theoretischen Güte um
\begin{equation*}
    \frac{Q_\text{theo}-Q_\text{exp}}{Q_\text{theo}} = \SI{10.492}{\percent}
\end{equation*}
ab, woraus sich schließen lässt, dass die Güte relativ genau gemessen werden konnte. \\
In der Tabelle \ref{tab:abweichen} sind die Suszeptibilitäten zusammengefasst mit den Abweichungen aufgelistet. 
\begin{table}
    \centering
    \caption{Experimentell bestimmte und theoretische Werte mit der dazugehörigen Abweichung}
    \label{tab:abweichen}
    \begin{tabular} {S[table-format=3.0] S[table-format=1.2] S[table-format=2.2] S[table-format=1.2] S[table-format=3.2] S[table-format=3.2]}
        \toprule
        {$\text{Probe}$} & {$\chi_R \cdot \num{e-2}$} & {$\chi_U \cdot \num{e-2}$} & {$\chi_\text{theo} \cdot \num{e-2}$} & 
        {$\eta_R \mathbin{/} \si{\percent}$} & {$\eta_U \mathbin{/} \si{\percent}$} \\
    \midrule
    {$\ce{Dy2O3}$} & 2.42 & 12.2  & 2.54 & 4.72 & 380.31 \\
    {$\ce{Nd2O3}$} & 0.61 & 0.71  & 0.25 & 144  & 184    \\
    \bottomrule
\end{tabular}
\end{table}
Bei der Betrachtung der Tabelle \ref{tab:abweichen} fällt auf, dass die mit der Spannungsdifferenz bestimmten Suszeptibilitäten stärker von dem 
theoretischen Wert abweichen, so dass die Vermutung nahe liegt, dass diese Methode ungenauer ist. 
In der Gleichung \eqref{eqn:chil} wird erkenntlich, dass dies  eine Annäherung für sehr große Frequenzen ist, so dass diese Annährung zu Ungenauigkeiten 
führen kann.
Ebenfalls sticht die experimentell bestimmte Suszeptibilität für $\ce{Nd2O3}$ heraus, da diese größere Abweichungen als die von $\ce{Dy2O3}$
besitzt.
Dies könnte der Tatsache geschuldet sein, dass die Spannungs- und Widerstandsdifferenzen deutlich geringer sind und das Auflösungsvermögen der Messgeräte 
nicht ausreicht, um solche Größenordnungen mit einer hohen Genauigkeit aufzunehmen.