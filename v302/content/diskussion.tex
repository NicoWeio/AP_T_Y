\section{Diskussion}
\label{sec:Diskussion}
Zuerst muss angemerkt werden, dass die Messwerte für die Versuche \ref{subsec:kapaaus}, \ref{subsec:induaus} und \ref{subsec:Maxwellaus} nicht selbst gemessen wurden.
Die Versuche funktionierten auch nach mehrfachen Anläufen nicht. Dabei wurden sowohl die Induktivitäten, Widerstände und Kapazitäten gewechselt.
Eine Ursache für das Problem ließ sich also nicht feststellen.
Fehler entstehen durch Defekekte der einzelnen Komponenten, so wie durch das Rauschen des Sinusgenerators.
Es folgen Diskussionen für die einzelnen Auswertungen.\\
Bei der Auswertung der Wheatstoneschen-Brückenschaltung fällt auf, dass bei der ersten Messung \ref{tab:Wheat} 
im Vergleich zur zweiten Messung \ref{tab:Wheatr} kleinere Fehler auftreten.
Eine Ursache könnte der höhere Wert bei dem zweiten unbekannten Widerstand sein. Dadurch wird nach der Gaußschen Fehlerfortpflanzung
der Fehler auch etwas größer.
Allgemein lässt sich sagen, dass die Wheatstonesche-Brückenschaltung sehr Präzise die Widerstände ermittelt.\\
Bei der Auswertung der Kapazitätsmessbrücke werden die unbekannten Kapazitäten sehr präzise bestimmt, während die Widerstände der Kondensatoren einen relativ hohen Fehler aufweisen.
Bei qualitativ hochwertigen Kondensatoren würde dies nicht von hoher Relevanz sein, da diese Widerstände gegen Null laufen würden.
Zur Ermittlung der Kapazitäten ist diese Brückenschaltung geeignet. \\
Die Fehler der Induktivitätsmessbrücke sind sehr ähnlich zu den Fehlern der Kapazitätsmessbrücke.
Die Bestimmung der Induktivitäten ist sehr präzise, während die Widerstände etwas ungenauer bestimmt werden.
Auch hier würde sich eine hochwertige Spule anbieten.\\
Die Maxwell-Brücke ist im Vergleich mit der Induktivitätsmessbrücke etwas unpräziser.
Dies liegt an den größeren Fehlern der Komponenten $R_3$ und $R_4$. 
Die Fehler der Widerstände sind äquivalent zu denen der Induktivitätsmessbrücke, wobei die Fehler der Induktivitäten dem fünffachen der
Induktivitätsmessbrücke entsprechen. Würden die Widerstände genauer sein, könnte diese Messbrücke die Induktivitäten ebenfalls präzise bestimmen, 
da nur ein Widerstand variiert werden muss.\\
Bei der frequenzabhängigen Messung mit der Wien-Robinson-Messbrücke ist die Differenz der gemessenen Frequenz $v_0$ und der berechneten Frequenz $w_0$ sehr groß.
Dies könnte durch den sehr großen Klirrfaktor begründet werden, wobei dieser durch Approximation der Oberwellen auch unpräzise ist.
Für weitere Annahmen müssten der Versuch mit verändeten Widerständen und Kapazitäten wiederholt werden.

