\section{Auswertung}
\label{sec:Auswertung}
\subsection{Wheatstonesche Brücke}
Mit der Wheatstoneschen Brückenschaltung sollten zwei Unbekannte Widerstände ermittelt werden.
In der folgenden Tabelle werden die Messwerte $R_2$ und $\frac{R_3}{R_4}$ der Messreihe aufgelistet. Zudem ist, der mit Hilfe von $\symbf{Zitat}$, berechnete 
Wert für $R_\text{x}$ eingetragen. Der Fehler lässt mit Hilfe der Gaußschen Fehlerfortpflanzung
\begin{equation}
  \sigma_{R_x}=\sqrt{\left(\frac{R_3}{R_4}\right)^2 \cdot \left(\symup{\Delta}R_2\right)^2 + R_2^2 \cdot \left(\symup{\Delta}\frac{R_3}{R_4}\right)^2}
\end{equation}
berechnen. Die Fehler sind gegeben durch $\symup{\Delta}R_2$ = $ 0.02 \cdot R_2$
und $\symup{\Delta}\frac{R_3}{R_4}$ = $ 0.05 \cdot \frac{R_3}{R_4}$.
%
%
%
\begin{table}
  \centering
  \label{tab:Wert10}
  \caption{Messwerte und berechnete Werte für Widerstand $R_\text{x}$ (Wert 10)}
  \sisetup{table-format = 4}
  \begin{tabular}{
    c
    S @{${}\pm{}$} S[table-format=2.2]
    S[table-format=1.2] @{${}\pm{}$} S[table-format=1.2]
    S[table-format = 3.2] @{${}\pm{}$} S[table-format=2.2]}
     \toprule
     {Wert 10}  &
     \multicolumn{2}{c}{$R_2 \mathbin{/} \si{\ohm}$}       &
     \multicolumn{2}{c}{$\frac{R_3}{R_4}$}       & 
     \multicolumn{2}{c} {$R_\text{x}  \mathbin{/} \si{\ohm}$}\\
     \cmidrule(lr){2-3} \cmidrule(lr){4-5} \cmidrule(lr){6-7}
     \midrule
     Messung 1 & 332  & 6.64  & 1.48 & 0.07 & 240.41 & 12.95\\
     Messung 2 & 664  & 13.28 & 0.74 & 0.04 & 238.17 & 12.82\\
     Messung 3 & 1000 & 20    & 0.49 & 0.02 & 239.93 & 12.92\\
      \bottomrule
  \end{tabular}
\end{table}%
%
Der Mittelwert:
%
\begin{table}
  \centering
  \label{tab:Wert11}
  \caption{Messwerte und berechnete Werte für Widerstand $R_\text{x}$ (Wert 11)}
  \sisetup{table-format = 4}
  \begin{tabular}{
    c
    S @{${}\pm{}$} S[table-format=2.2]
    S[table-format=1.2] @{${}\pm{}$} S[table-format=1.2]
    S[table-format = 3.2] @{${}\pm{}$} S[table-format=2.2]}
     \toprule
     {Wert 11}  &
     \multicolumn{2}{c}{$R_2 \mathbin{/} \si{\ohm}$}       &
     \multicolumn{2}{c}{$\frac{R_3}{R_4}$}       & 
     \multicolumn{2}{c} {$R_\text{x}  \mathbin{/} \si{\ohm}$}\\
     \cmidrule(lr){2-3} \cmidrule(lr){4-5} \cmidrule(lr){6-7}
     \midrule
     Messung 1 & 332  & 6.64  & 0.72 & 0.04 & 491.82 & 26.49\\
     Messung 2 & 664  & 13.28 & 0.35 & 0.02 & 488.78 & 26.32\\
     Messung 3 & 1000 & 20    & 0.24 & 0.01 & 492.54 & 26.52\\
      \bottomrule
  \end{tabular}
\end{table}
Der Mittelwert:
\subsection{Kapazitätsmessbrücke}
