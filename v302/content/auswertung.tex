\section{Auswertung}
\label{sec:Auswertung}
\subsection{Wheatstonesche Brücke}
\label{sec:Wheat}
Mit der Wheatstoneschen Brückenschaltung sollten zwei Unbekannte Widerstände ermittelt werden.
In der folgenden Tabelle werden die Messwerte $R_2$ und $\frac{R_3}{R_4}$ der Messreihe aufgelistet. Zudem ist, der mit Hilfe von $\symbf{Zitat}$, berechnete 
Wert für $R_\text{x}$ eingetragen. Der Fehler lässt mit Hilfe der Gaußschen Fehlerfortpflanzung
\begin{equation}
  \label{eqn:fehR}
  \sigma_{R_x}=\sqrt{\left(\frac{R_3}{R_4}\right)^2 \cdot \left(\symup{\Delta}R_2\right)^2 + R_2^2 \cdot \left(\symup{\Delta}\frac{R_3}{R_4}\right)^2}
\end{equation}
berechnen. Die Fehler sind gegeben durch $\symup{\Delta}R_2$ = $ 0.02 \cdot R_2$
und $\symup{\Delta}\frac{R_3}{R_4}$ = $ 0.05 \cdot \frac{R_3}{R_4}$.
%
%
%
\begin{table}
  \centering
  \label{tab:Wert10}
  \caption{Messwerte und berechnete Werte für Widerstand $R_\text{x}$ (Wert 10)}
  \sisetup{table-format = 4}
  \begin{tabular}{
    c
    S @{${}\pm{}$} S[table-format=2.2]
    S[table-format=1.2] @{${}\pm{}$} S[table-format=1.2]
    S[table-format = 3.2] @{${}\pm{}$} S[table-format=2.2]}
     \toprule
     {Wert 10}  &
     \multicolumn{2}{c}{$R_2 \mathbin{/} \si{\ohm}$}       &
     \multicolumn{2}{c}{$\frac{R_3}{R_4}$}       & 
     \multicolumn{2}{c} {$R_\text{x}  \mathbin{/} \si{\ohm}$}\\
     \cmidrule(lr){2-3} \cmidrule(lr){4-5} \cmidrule(lr){6-7}
     \midrule
     Messung 1 & 332  & 6.64  & 1.48 & 0.07 & 240.41 & 12.95\\
     Messung 2 & 664  & 13.28 & 0.74 & 0.04 & 238.17 & 12.82\\
     Messung 3 & 1000 & 20    & 0.49 & 0.02 & 239.93 & 12.92\\
      \bottomrule
  \end{tabular}
\end{table}%
\\
Der Mittelwert der $R_\text{x}$ lässt sich berechnen mit 
\begin{equation}
  \label{eqn:mit}
  \bar{R_\text{x}}=\sum_{i=1}^3 \frac{1}{3}R_{x_i}\, .
\end{equation}
Daraus folgt für den Mittelwert des Widerstands Wert 10, $R_\text{x}= \SI{240\pm7}{\ohm}$.
%
\begin{table}
  \centering
  \label{tab:Wert11}
  \caption{Messwerte und berechnete Werte für Widerstand $R_\text{x}$ (Wert 11)}
  \sisetup{table-format = 4}
  \begin{tabular}{
    c
    S @{${}\pm{}$} S[table-format=2.2]
    S[table-format=1.2] @{${}\pm{}$} S[table-format=1.2]
    S[table-format = 3.2] @{${}\pm{}$} S[table-format=2.2]}
     \toprule
     {Wert 11}  &
     \multicolumn{2}{c}{$R_2 \mathbin{/} \si{\ohm}$}       &
     \multicolumn{2}{c}{$\frac{R_3}{R_4}$}       & 
     \multicolumn{2}{c} {$R_\text{x}  \mathbin{/} \si{\ohm}$}\\
     \cmidrule(lr){2-3} \cmidrule(lr){4-5} \cmidrule(lr){6-7}
     \midrule
     Messung 1 & 332  & 6.64  & 0.72 & 0.04 & 491.82 & 26.49\\
     Messung 2 & 664  & 13.28 & 0.35 & 0.02 & 488.78 & 26.32\\
     Messung 3 & 1000 & 20    & 0.24 & 0.01 & 492.54 & 26.52\\
      \bottomrule
  \end{tabular}
\end{table}
\\
Mit der Formel für den Mittelwert \ref{eqn:mit} folgt für Wert 11, $R_\text{x}=\SI{491\pm15}{\ohm}$
\newpage%
%
%
%
\subsection{Kapazitätsmessbrücke}
In der Messung um den realen Kondensator, ist dieser durch einen ohmschen Widerstand und eine Widerstandslose Kapazität ersetzt worden.
Zum berechnen der Werte wurde die Gleichung $\symbf{Referenz}$ genutzt.
Alle Messwerte, berechneten Kapazitäten und Widerstände für den Wert 8 sind in der Tabelle \ref{tab:Wert8} aufgeführt.
Dabei wurden die Fehler des Widerstands mit der Formel \ref{eqn:fehR} und
die der Kapazität mit 
\begin{equation}
  \symup{\Delta}C_x=\sqrt{\left(\frac{R4}{R3}\right)^2 \cdot \left(\symup{\Delta}C_2\right)^2
  +\left(-C_2\frac{R4}{R3}\right)^2 \cdot \left(\symup{\Delta}\frac{R4}{R3}\right)^2}
\end{equation}
berechnet. Wobei $\symup{\Delta}R_2=0.03 \cdot R_2$ der Fehler von dem Variablen Widerstand ist und $\symup{\Delta}C_2=0.02 \cdot C_2$ der Fehler 
der Kapazität ist. Die Abweichung von $\frac{R_3}{R_4}$ ist die selbe wie in Abschnitt \ref{sec:Wheat}.
\begin{table}
  \centering
  \label{tab:Wert8}
  \caption{Messwerte und berechnete Werte für realen Kondensator, 
  $R_\text{x}$ und $C_\text{x}$ (Wert 8)}
  \sisetup{table-format = 3}
  \begin{tabular}{
    c
    S @{${}\pm{}$} S[table-format=2.2]
    S[table-format=1.2] @{${}\pm{}$} S[table-format=1.2]
    S @{${}\pm{}$} S[table-format=2.2]
    S[table-format=3.2] @{${}\pm{}$} S[table-format=2.2]
    S[table-format = 3.2] @{${}\pm{}$} S[table-format=2.2]}
     \toprule
     {Wert 8}  &
     \multicolumn{2}{c}{$R_2 \mathbin{/} \si{\ohm}$}       &
     \multicolumn{2}{c}{$\frac{R_3}{R_4}$}                 & 
     \multicolumn{2}{c}{$C_2 \mathbin{/} \si{\nano\farad}$} &
     \multicolumn{2}{c}{$R_\text{x} \mathbin{/} \si{\ohm}$}&
     \multicolumn{2}{c} {$C_\text{x}  \mathbin{/} \si{\nano\farad}$}\\
     \cmidrule(lr){2-3} \cmidrule(lr){4-5} \cmidrule(lr){6-7}
     \midrule 
     Messung 1 & 371  & 11.13  & 1.54 & 0.08 & 450 & 9    & 570.62& 33.27 & 292.57 & 15.76\\
     Messung 2 & 418  & 12.54  & 1.37 & 0.07 & 399 & 7.89 & 572.52& 33.38 & 291.31 & 15.69\\
     Messung 3 & 278  & 8.34   & 2.06 & 0.1  & 597 & 11.94& 572.15& 33.36 & 290.07 & 15.62\\
      \bottomrule
  \end{tabular}
\end{table}
\\
Der Mittelwert wird ermittelt mit \ref{eqn:mit} und mit \begin{equation}
  \label{eqn:mit2}
  \bar{C_\text{x}}=\sum_{i=1}^3 \frac{1}{3}C_{x_i}\, .
\end{equation}
Somit bekommt man $C_\text{x}=\SI{291 \pm 9}{\nano\farad}$ und $R_\text{x}=\SI{572 \pm 19}{\ohm}$ für den Kondensator mit Wert 8.
%
\\
Äquivalent dazu wurden für den zweiten Unbekannten Kondensator die Werte berechnet und in Tabelle \ref{tab:Wert9} eingetragen.
\begin{table}
  \centering
  \label{tab:Wert9}
  \caption{Messwerte und berechnete Werte für realen Kondensator,
   $R_\text{x}$ und $C_\text{x}$ (Wert 9)}
  \sisetup{table-format = 3}
  \begin{tabular}{
    c
    S @{${}\pm{}$} S[table-format=2.2]
    S[table-format=1.2] @{${}\pm{}$} S[table-format=1.2]
    S @{${}\pm{}$} S[table-format=2.2]
    S[table-format=3.2] @{${}\pm{}$} S[table-format=2.2]
    S[table-format = 3.2] @{${}\pm{}$} S[table-format=2.2]}
     \toprule
     {Wert 9}  &
     \multicolumn{2}{c}{$R_2 \mathbin{/} \si{\ohm}$}       &
     \multicolumn{2}{c}{$\frac{R_3}{R_4}$}                 & 
     \multicolumn{2}{c}{$C_2 \mathbin{/} \si{\nano\farad}$} &
     \multicolumn{2}{c}{$R_\text{x} \mathbin{/} \si{\ohm}$}&
     \multicolumn{2}{c} {$C_\text{x}  \mathbin{/} \si{\nano\farad}$}\\
     \cmidrule(lr){2-3} \cmidrule(lr){4-5} \cmidrule(lr){6-7}
     \midrule 
     Messung 1 & 466  & 13.98  & 1.04 & 0.08 & 450 & 9    & 486.97 & 28.39 & 430.63 & 23.19\\
     Messung 2 & 524  & 15.72  & 0.93 & 0.07 & 399 & 7.89 & 485.63 & 28.32 & 430.52 & 23.18\\
     Messung 3 & 352  & 10.56  & 1.39 & 0.1  & 597 & 11.94& 488.1  & 28.46 & 430.54 & 23.19\\
      \bottomrule
  \end{tabular}
\end{table}
\\
Entsprechend der Rechnung für Wert 8, werden die Mittelwerte für die Werte von Kondensator Wert 9 errechnet.
Es resultiert für die Kapazität $C_\text{x}=\SI{431 \pm 13}{\nano\farad}$ und
 für den Widerstand des Kondensators $R_\text{x}=\SI{487 \pm 16}{\ohm}$ .

 \newpage
  \subsection{Induktivitätsbrücke}
