\section{Durchführung}
\label{sec:Durchführung}
Um die Abgleichung durchführen zu können, wurde als Nullindikator ein Oszilloskop verwendet.
\subsection{Frequenzunabhängige Messung}
Theoretisch müssten sich die frequenzunabhängigen Brücken bei jeder beliebigen Frequenz abgleichen lassen. Jedoch ist dies technisch nicht realisierbar,
so dass die Streukapazitäten bei zu hohen Frequenzen zu stark werden und der Abgleichvorgang bei zu niedrigen Frequenzen zu lange dauert.   
Die Frequenz, bei welcher die statische Messung gut funktioniert, liegt bei  $ \omega = \SI{1}{\kilo\hertz}$, so dass der Versuch mit dieser Frequenz durchgeführt wurde.\\
Zunächst wird ein Ohmscher Widerstand mit der Wheatstoneschen Brückenschaltung ausgemssen. Um die Brückenschaltung abzugleichen wird ein 
Zehngang-Präzisions-Potentiometer  mit $\SI{1}{\kilo\ohm}$ verwendet. Somit kann das Verhältnis $\sfrac{R_3}{R_4}$ varriert werden.
Nach dem Zusammenbau wird die Brückenschaltung an eine Wechselstromquelle angeschlossen. Fortlaufend wird das Verhältnis 
$\sfrac{R_3}{R_4}$ so lange varriert, bis das Oszilloskop bis auf unvermeidbare 
Oszilaltionen keine Spannung mehr anzeigt. Zur Fehlerbestimmung muss der Widerstand $R_2$ gewechselt werden.
Dieser Prozess wird für zwei unbekannte Widerstände durchgeführt.\\
Bei der Kapazitätsmessung werden die Widerstände $R_2$ und $\sfrac{R_3}{R_4}$ alternierend gewählt. Mit Hilfe des Verhätnisses 
$\sfrac{R_3}{R_4}$ wird die Brückenspannung minimiert. Danach wird die Brückenspannung erneut mit $R_2$ verkleinert. Anschließend 
wird die Brücke mit $\sfrac{R_3}{R_4}$ erneut abgeglichen. Dieses Wechselspiel wird bis zum absoluten Spannungsminimum von $U_B$ durchgeführt.
Zur Fehlerbestimmung muss die Kapazität $C_2$ gewechselt werden. Auch hier werden zwei unbekannte Kapazitäten ausgemessen.\\
Bei der ersten Induktivitätsmessung werden einerseits eine Induktivitätsmessbrücke und eine Maxwell-Brücke eingesetzt. 
Hierzu gleicht man die Brücken mit der Abgleichbedingung ab und misst die selbe unbekannte Induktivität und den selben Verlustwiderstand aus.\\
\subsection{Frequenzabhängige Messung}
Hierbei wird sich die Wien-Robinson-Brücke zu Nutze gemacht. Zunächst wird der Frequenzbereich zwischen $\SI{20}{\hertz}$ bis $\SI{30000}{\hertz}$
abgetastet. Dazu werden $U_B\left(\omega\right)$ und $U_S\left(\omega\right)$ gemessen. Wichtig hierbei ist die Frequenz $\omega_0$, bei welcher die Brückenspannung
minimal ist.