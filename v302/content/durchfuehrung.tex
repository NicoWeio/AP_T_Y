\section{Durchführung}
\label{sec:Durchführung}
Um die Abgleichung durchführen zu können, wurde als Nullindikator ein Oszilloskop verwendet.
\subsection{Frequenzunabhängige Messung}
Die Frequenz, bei welcher die statische Messung gut funktioniert, liegt bei  $ \omega = \SI{1000}{\per\second}$, so dass der Versuch mit dieser Frequenz durchgeführt wurde.\\
Zunächst wird ein Ohmscher Widerstand mit der Wheatstoneschen Brückenschaltung ausgemssen. Um die Brückenschaltung abzugleichen wird ein 
Zehngang-Präzisions-Potentiometer  mit $\SI{1}{\kilo\ohm}$ verwendet. Somit kann das Verhältnis $\sfrac{R_3}{R_4}$ varriert werden.
Nach dem Zusammenbau wird die Brückenschaltung an eine Wechselstromquelle angeschlossen. Fortlaufend wird das Verhältnis 
$\sfrac{R_3}{R_4}$ so lange varriert, bis das Oszilloskop bis auf unvermeidbare 
Streufrequenzen keine Spannung mehr anzeigt. Zur Fehlerbestimmung muss der Widerstand $R_2$ gewechselt werden.
Dieser Prozess wird für zwei unbekannte Widerstände durchgeführt.\\
Bei der Kapaitätsmessung werden die Widerstände $R_2$ und $\sfrac{R_3}{R_4}$ aternierend gewählt. Mithilfe des Verhätnises 
$\sfrac{R_3}{R_4}$ minimiert wird die Brückenspannung minimiert. Danach wird die Brückenspannung erneut mit $R_2$ verkleinert. Anschließend 
wird mit $\sfrac{R_3}{R_4}$ die Brücke erneut abgeglichen. Dieses Wechselspiel wird bis zum abosluten Spannnungsminimum von $U_B$ durchgeführt.
Zur Fehlerbestimmung muss die Kapazität $C_2$ gewechselt werden. Auch hier werden zwei unbekannte Kapazitäten ausgemessen.\\
Bei der ersten Induktivitätsmessung werden einerseits eine Induktivitätsmessbrücke und eine Maxwell-Brücke eingesetzt. 
Hierzu gleicht man die Brücken mit der Abgleichbedingung ab und misst die selbe unbekannte Induktivität und den selben Verlustwiderstand aus.\\
\subsection{Frequenzabhängige Messung}
Hierbei wird sich die Wien-Robinson-Brücke zu Nutze gemacht. Zunächst wird der Frequenzbereich zwischen $\SI{20}{\hertz}$ bis $\SI{30000}{\hertz}$
abgetastet. Dazu werden $U_B\left(\omega\right)$ und $U_S\left(\omega\right)$ gemessen. Wichtig hierbei ist die Frequenz $\omega_0$, bei welcher die Brückenspannung
minimal ist.