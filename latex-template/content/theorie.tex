\section{Theorie}
\label{sec:Theorie}
Ohne äußere Einflüsse ist die Richtung des Wärmeflusses immer vom warmen Reservoir in das kalte Reservoir.
Durch anwenden von mechanischer Arbeit $A$ lässt sich dieser Prozess auch in die andere Richtung durchführen.
\subsection{Das Prinzip der Wärmepumpe}
Nach dem ersten Hauptsatz der Thermodynamik beträgt die an das wärmere Medium abgegebene Wärmemenge $Q_1$, die summe der vom kühleren Medium aufgenommenen
Wärmeenergie $Q_2$ und der Arbeit $A$. Also gilt
\begin{equation}
    Q_1=Q_2+A
    \label{eqn:H1}
\end{equation}
Die Güteziffer $\nu$ der Wärmepumpe beschreibt das Verhältnis der abgegebenen
Wärmemenge $Q_1$ zur aufgewandten Arbeit $A$
\begin{equation}
    \nu=\frac{Q_1}{A}
    \label{eqn:Güte}
\end{equation}
Der zweite Hauptsatz der Thermodynamik führt für die reduzierten Wärmemengen 
zu der Beziehung, dass deren Summe $\int \frac{\symup{d}Q}{T}$ null beträgt. 
Aus dieser folgt
\begin{equation}
    \frac{Q_1}{T_1} - \frac{Q_2}{T_2} = 0
    \label{eqn:redWärm}
\end{equation}
Jedoch gilt dies nur bei Idealen reversiblen Prozessen. In der technischen Anwendung gilt
\begin{equation}
    \frac{Q_1}{T_1} - \frac{Q_2}{T_2} > 0
    \label{eqn:ungWärme}
\end{equation}
aus \ref{eqn:H1} und \ref{eqn:redWärm} folgt
\begin{equation*}
    Q_1 = A + \frac{T_2}{T_1} \cdot Q_1 \; \text{,}
\end{equation*}
aus der Formel für die Güteziffer\ref{eqn:Güte}, für einen reversiblen Vorgang, folgt die Güteziffer einer Idealen Wärmepumpe
\begin{equation}
    \nu_\text{id} = \frac{Q_1}{A} = \frac{T_1}{T_1 - T_2}
    \label{eqn:Id}
\end{equation}
Für die reale Wärmepumpe folgt aus \ref{eqn:H1} und \ref{eqn:ungWärme}
\begin{equation}
    \nu_\text{re} < \frac{T_1}{T_1 - T_2} \; \text{.}
    \label{eqn:Re}
\end{equation}
Aus den Gleichungen \ref{eqn:Id} und \ref{eqn:Re} folgt, dass eine Wärmepumpe effizienter Arbeitet, je kleiner die Temperaturdifferenz
zwischen den beiden WärmeReservoirs ist. Der Vorteil einer Wärmepumpe liegt außerdem in der möglichkeit eine Wärmemenge $Q_2$, welche frei Verfügbar ist 
auszunutzen um Preisgünstig $Q_1$ zu heizen. Damit kann die verrichtete Arbeit, unter günstigen Bedingungen erheblich kleiner sein als die gewonnene Wärmemenge $Q_1$.
Damit folgt das eine Wärmepumpe gegenüber einem Wärmegewinnungsverfahren, welches machanische Energie direkt in Wärme umwandelt, effizienter ist.
Hier ist die erhaltene Wärmemenge höchstens gleich der mechanischen Arbeit, also
\begin{equation*}
    Q_{1_\text{direkt}}\leq A
\end{equation*}
im Gegensatz zu
\begin{equation*}
    Q_{1_\text{rev}}=A\,\frac{T_1}{T_1-T_2}
\end{equation*}.



\subsection{Die Arbeitsweise einer Wärmepumpe}
\begin{figure}
    \centering
    \includegraphics[scale=0.4]{aufbau.pdf}
    \caption{Aufbau einer Wärmepumpe [1]}
    \label{fig:aufbau}
\end{figure}
Als Transportmedium in der Wärmepumpe fungiert ein reales Gas,
welches bei Wärmeaufnahme verdampft und die Wärme durch Kondensation wieder abgibt. Der Wärmetransport wird also in Form 
von Phasenumwandlungsenergie des Gases erreicht.
Die Wahl des Gases hängt von der Kondensationswärme ab. Günstig ist es eine möglichst hohe Kondensationswärme zu haben. \\
In Abbbildung \ref{fig:aufbau} ist der schematische Aufbau der Wärmepumpe dargestellt.
Demnach sorgt der Kompressor K für einen Mediumkreislauf, durch den das Transportgas beide WärmeReservoirs und das Drosselventil D durchläuft.
An diesem entsteht ein Druckunterschied $p_2-p_1$\footnote{$p_\text{a}=p_1, p_\text{b}=p_2$}. Dabei ist das Transportgas in Reservoir 1 unter dem Druck $p_2$
und der Temperatur $T_1$ flüssig und bei $p_2$ und $T_2$ gasförmig.\\
Dem kälteren Reservoir 2 wird durch das Verdampfen des Mediums die Verdampfungswärme $L$ pro gramm entzogen. 
Anschließend wird das Gas im Kompressor adiabatisch komprimiert, wobei es sich stark erwärmt und der Druck soweit ansteigt, bis es sich im Reservoir 1
wieder verflüssigt. Dabei gibt das Gas die Kondensationswärme $L$ pro Gramm an 1 ab und heizt dieses auf.\\
Weitere nötige Komponenten der Wärmepumpe sind ein Reiniger R, welcher das Medium von Gasresten trennt bevor es uim Drosselventil gelangt, sowie ein Steuerelement S,
welches mit dem Drosselventil D gekoppelt ist und das schädliche Eindringen flüssigen Mediums in den Kompressor über Kontrolle der Temperaturdifferenz
am Ein-und Ausgang des zweiten Reservoirs verhindert. 
\subsection{Die Bestimmung der Kenngrößen einer realen Wärmepumpe}
\begin{figure}
    \centering
    \includegraphics[scale=0.5]{aufbau2.pdf}
    \caption{Aufbau der Messapparatur [1]}
    \label{fig:aufbau2}
\end{figure}
Folgende Kenngrößen sind bei einer realen Wärmepumpe relevant: die Güteziffer, der Massendurchsatz dm/dt des Transportmediums und der Wirkungsgrad des Kompressors. Die Kenngrößen kann man errechnen mit den 
Ergebnissen einer Messreihe, die man mit der in \ref{fig:aufbau2} dargestellten Messapperatur bekommt.\\
Für die Zeitabhängige Messungen der Drücke $p_1$, $p_2$ gibt es zwei zwischen den Reservoirs und dem Drosselventil D installierte Manometer. Die 
Temperaturverläufe werden mit zwei digitalen Thermometern in den Reservoirs gemessen und für die Leistungsaufnahme ist ein Wattmeter an den Motor des Kompressors geschaltet.\\
Die Reservoirs sind in thermisch isolierten Gefäßen und werden während der Messung von zwei Rührmotoren umgerührt.\\
\subsubsection{Die reale Güteziffer}
Mit der Messung der Temperatur $T_1$ in Abhängigkeit von der Zeit $t$ lässt sich die reale Güteziffer als 
\begin{equation}
    \nu = (m_1 c_\text{w} + m_\text{k} c_\text{k})\cdot 
    \frac{\symup{\Delta} T_1}{\symup{\Delta} t} \cdot \frac{1}{N}
    \label{eqn:Güteziffer}
\end{equation}
berechnen.\\
Dabei ist $m_1$ die Wassermasse und $c_\text{w}$ deren Wärmekapazität. Die Masse der Kupferschlange und des Eimers werden durch $m_\text{k}$ beschrieben und deren Wärmekapazität
durch $c_\text{k}$. Die am Wattmeter abgelesenen Werte $N$, geben über das Zeitintervall $\increment t$ gemittelte Leistungsaufnahme des Kompressors an.

\subsubsection{Der Massendurchsatz}
Mit der Messung der Temperatur $T_2$ lässt sich, bei bekannter Verdampfungswärme $L$, der Massenduchsatz durch\\
\begin{equation}
    \frac{\symup{d} m}{\symup{d}t} = (m_2 c_\text{w} + m_\text{k} c_\text{k}) \cdot 
    \frac{\symup{d} T_2}{\symup{d}t} \cdot  \frac{1}{L}
    \label{eqn:Massendurchsatz}
\end{equation}
bestimmen.\\
Mit der Wassermasse $m_2$ des Reservoirs 2.\\
Zur Bestimmung des Massenduchsatzes wird außerdem die Verdampfungswärme $L$ benötigt.
Berechnen lässt sich diese mithilfe der Clausius-Clapeyron'schen Gleichung.

\subsubsection{Die mechanische Kompressorleistung}
Bei der Komprimierung eines Gasvolumens $V_\text{a}$ zu dem Volumen $V_\text{b}$ 
verrichtet der Kompressor die Arbeit
\begin{equation}
    A = - \int^{V_\text{b}}_{V_\text{a}} p \; \symup{d}V \; \; \text{.}
\end{equation}
Für die adiabatische komprimierung gilt die Poissonsche Gleichung
\begin{equation}
    p_\text{a} V^\kappa_\text{a} = p_\text{b} V^\kappa_\text{b} = p V^\kappa \; \;
    \text{,}
\end{equation}
Damit folgt für $A$
\begin{equation}
    \begin{split}
        A &= - p_\text{a} V^\kappa_\text{a} \int^{V_\text{b}}_{V_\text{a}} V^{-\kappa}
        \; \symup{d}V = \frac{1}{\kappa - 1} p_\text{a} V^\kappa_\text{a} \left(
        V^{-\kappa + 1}_\text{b} - V^{-\kappa + 1}_\text{a} \right) \\
        &= \frac{1}{\kappa - 1} \left(p_\text{b} \sqrt[\kappa]{
        \frac{p_\text{a}}{p_\text{b}}} - p_\text{a}\right) V_\text{a}
    \end{split}
\end{equation}
und für die mechanische Kompressionsleistung
\begin{equation}
    \begin{split}
        N_\text{mech} &= \frac{\symup{d}A}{\symup{d}t} = \frac{1}{\kappa - 1}
        \left(p_\text{b} \sqrt[\kappa]{ \frac{p_\text{a}}{p_\text{b}}} 
        - p_\text{a} \right) \frac{\symup{d}V_\text{a}}{\symup{d}t} \\
        &= \frac{1}{\kappa - 1} \left(p_\text{b} \sqrt[\kappa]{ 
        \frac{p_\text{a}}{p_\text{b}}} - p_\text{a}\right) \frac{1}{\rho}
        \frac{\symup{d}m}{\symup{d}t}
        \label{eqn:Kompressor}
    \end{split}
\end{equation}
mit $\rho$ als die Dichte des Transportmediums im gasförmigen Zustand unter dem Druck $p_1$.
Mit hilfe der idealen Gasgleichung lässt sich $\rho$ unter den Normalbedingungen 
$p = \SI{1}{\bar}$ und $T = \SI{0}{\celsius}$ bestimmen.