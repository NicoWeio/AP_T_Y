\section{Diskussion}
\label{sec:Diskussion}
In der Abbildung \ref{fig:tiefdruck} wird ersichtlich, dass die Messwerte eine relativ hohe Abweichung aufweisen.
Diese Beobachtung wird durch die hohen Unsicherheiten der Regressionsparameter gestärkt.
Gründe hierfür könnte die gleichzeitige Messung von drei Messwerten sein, welche der Druck, die Temperatur des flüssigen Zustands und die Temperatur des Gases 
waren.
Da diese nicht gleichzeitig abgelesen und notiert wurden, sondern nur nacheinander abgelesen wurden, konnte eine der Größen, während eine andere Größe gemessen wurde,
weiter steigen.
Ein weiterer Faktor könnten die Theromstate sein, da die Ableseganaugkeit relativ gering ist, da dort per Augenmaß erfasst werden muss, welche Temperatur angezeigt wird.
Ebenfalls bieten die Theromstate kein allzu hohes Auflösungsvermögen, was weniger Präzision bietet. 