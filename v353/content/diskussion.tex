\section{Diskussion}
\label{sec:Diskussion}
In Abblidung \ref{fig:discharge} fällt direkt auf, dass dort zwei Ausgleichgeraden angelegt wurden. Der Grund dazu wurde schon in der Auswertung genannt.
Der Grund für diese starke Abweichung des letzten Messpunktes könnte an einem systematischen Fehler liegen. Einerseits wurde diese Messreihe vor einem Kabelaustausch
aufgenommen. Da das ausgetauschte Kabel nach der Messreihe einen Defekt aufwies, könnte dieser Defekt aber schon auf die vorige Messung Einfluss genommen haben, so dass
niedrige Spannungen besonders unscharf gemessen werden könnten.
Anderseits wurde die Funktion, die Zeitskala an dem Oszilloskop zu verringern, erst in der zweiten Messreihe entdeckt. Darunter könnte die Ablesgenauigkeit der ersten Messreihe
gelitten haben.
Jedoch wird die hohe Übereinstimmung des zweiten und dritten Messverfahrens auffällig. Der erste Wert für die Zeitkonstante weicht von dem zweiten Wert
um $\SI{59.53}{\percent}$ ab. Der zweite und der dritte Wert weisen nur eine Abweichung von $\SI{13.41}{\percent}$ auf. Dies könnte mit dem oben genannten
Fehler in Verbindung stehen.
Diese starke Abweichung könnte sich auch auf die Frequenzabhängigkeit bzw. Frequenzunabhängigkeit zurück führen lassen. 
Bei den frequenzabhängigen Messungen hat der Generatorinnenwiderstand von $R = \SI{600}{\ohm}$ einen Einfluss.
Ebenfalls anzumerken ist die Funktion des RC-Kreises als Integrator. 
Mit Hilfe des RC-Kreises lässt sich z.B. die Beziehung zwischen der Sinus- und Cosinus-Kurve sehr schön darstellen.