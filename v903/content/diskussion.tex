\section{Diskussion}
\label{sec:Diskussion}
Die errechneten Strömungsgeschwindigkeiten in dem Abschnitt \ref{sub:velocity} weisen eine klare Tendenz auf.
Je kleiner der Rohrdurchmesser ist desto schneller fließt die Dopplerflüssigkeit durch das Rohr.
Jedoch lässt sich erkennen, dass die errechneten Geschwindigkeiten zu den jeweiligen Rohren zwischen den jeweiligen
Winkeln abweicht.
Dies sollte in der Theorie nicht der Fall sein, da die Geschwindigkeit unabhängig von dem Prismawinkel ist.
Schon während der Durchführung wurde es ersichtlich, dass die Messung relativ ungenau ist bzw. die Messaperatur sehr empfindlich ist,
was diese Abweichungen verursachen könnte.
Die hohen Ungenauigkeiten werden in dem zweiten Abschnitt \ref{sec:profile} der Auswertung jedoch deutlich größer.
Dort lassen sich sehr viele Nulleinträge erkennen, weshalb eine Struktur des Strömungsprofils nicht ersichtlich wird.
Da dort eine hohe Priorität auf die richtigen Messtiefen gelegt wurde, lässt sich dort ein systematischer Fehler
mit einer hohen Wahrscheinlichkeit ausschließen. 
Jedoch fiel es schon während der Messung auf, dass die Frequenzverschiebungen schon nach einer Distanzerhöhung von $\symup{\Delta} d = \SI{0.5}{\micro\second}$
sprunghaft auf einen Wert oder auf 0 an- bzw. abstiegen. 
Somit steht die Vermutung im Raum, dass nach solch einer Veränderung der Distanz außerhalb des Rohrs gemessen wurde.