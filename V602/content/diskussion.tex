\section{Diskussion}
\label{sec:Diskussion}
In der der Tabelle \ref{tab:vgllit} sind die Literaturwerte für die Abschirmkonstanten $\sigma_{K\text{, Lit}}$ die experimentell 
bestimmten Abschirmkonstanten $\sigma_K$ und die daraus errechneten Verhältnisse $\eta$ aufgetragen.
\begin{table}
    \centering
    \caption{Vergleich der berechneten Abschirmkonstanten mit den Literaturwerten}
    \label{tab:vgllit}
    \begin{tabular} {S[table-format=2] S[table-format=1.3]  S[table-format=1.3] S[table-format=3.2]}
        \toprule
        & {$\sigma_{K\text{, Lit}}$} & {$\sigma_K$} & {$\eta \mathbin{/} \si{\percent}$} \\
    \midrule
    {Zn} & 3.566 & 3.668 & 102.85 \\
    {Ga} & 3.677 & 3.673 & 99.90 \\
    {Br} & 3.848 & 3.894 & 101.19 \\
    {Rb} & 3.944 & 4.037 & 102.36 \\
    {Sr} & 3.999 & 4.041 & 101.05 \\
    {Zr} & 4.101 & 4.28  & 104.37 \\
    \bottomrule
    \end{tabular}
\end{table}
Nach Betrachtung der Tabelle \ref{tab:vgllit} fällt auf, dass alle berechneten Werte etwas größer als die Literaturwerte sind.
Somit lässt sich vermuten, dass die Ursache ein systematischer Fehler ist.
Das Verhältnis der errechneten Rydbergenergie und des Literaturwertes beträgt
\begin{equation}
    \mu = \frac{R_\infty}{R_{\infty_\text{Lit}}} = \frac{\SI{12.74}{\electronvolt}}{\SI{13.606}{\electronvolt}} = \SI{93.63}{\percent} \; \text{.}
\end{equation}
Dies entspricht einer Abweichung von  etwa $\SI{7}{\percent}$, was im Vergleich zu der oben stehenden Tabelle \ref{tab:vgllit} eine etwas erhöhte Abweichung ist.
Dies könnte an Rundung von Dezimalstellen liegen. Anderseits könnte die Ursache bei der Abweichung der Absorptionsenergien liegen, da aus diesen die Rydbergfrequenz und somit 
auch die Rydbergenergie bestimmt  worden ist.