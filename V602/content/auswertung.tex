\section{Auswertung}
\label{sec:Auswertung}
Jegliche Fehlerrechnung wurde mit der python-Bibliothek uncertainties \cite{uncertainties} absolviert.
Trotz dessen sind die Formeln für die Unsicherheiten in den jeweiligen Abschnitten angegeben.
Allgemeine Rechnungen wurden mit der python-Bibliothek numpy \cite{numpy} automatisiert. 
Die graphischen Unterstützungen wurden mit Hilfe der python-Bibliothek matplotlib \cite{matplotlib} erstellt.
\subsection{Vorbereitungsaufgabe}
In der Tabelle \ref{tab:hallo} sind die recherchierten Kernladungszahlen $Z$, die Energien der K-Absorptionskanten $E_K$ und die daraus 
berechneten Winkel $\theta$ und die Abschirmkonstanten 
$\sigma_K$ für die jeweiligen Materialien gegeben.
\begin{table}
    \centering
    \caption{Literatur- und berechnete Werte für die Vorbereitungsaufgabe}
    \label{tab:hallo}
    \begin{tabular} {S[table-format=2.0] S[table-format=2.0]  S[table-format=2.2] S[table-format=2.1] S[table-format=1.3]}
        \toprule
        & {$Z$} & {$E_K \mathbin{/} \si{\kilo\electronvolt}$} & {$\theta \mathbin{/} \si{\degree}$} & {$\sigma_K \mathbin{/}$} \\
    \midrule    
    {Zn} & 30 & 9.65    & 18.6  & 3.566\\
    {Ga} & 31 & 10.37   & 17.3  & 3.61\\
    {Br} & 35 & 13.47   & 13.2  & 3.848\\
    {Rb} & 37 & 15.2    & 11.7  & 3.944\\
    {Sr} & 38 & 16.1    & 11    & 3.999\\
    {Zr} & 40 & 17.99   & 9.9   & 4.101\\
    {Ge} & 32 & 11.1    & 16.1  & 3.677\\
    \bottomrule
    \end{tabular}
\end{table}
\subsection{Überprüfung der Bragg-Bedingung} \label{sec:braggpr}
Zunächst muss die Bragg-Bedingung überprüft werden. 
Die dazu gemessenen Daten sind in der Tabelle \ref{tab:Bragg} aufgetragen. 
Außerdem sind die Werte in der Abbildung \ref{fig:Bragg} graphisch dargestellt.
\begin{table}
    \centering
    \caption{Gemessene Winkel und Impulse zur Überprüfung der Bragg-Bedingung}
    \label{tab:Bragg}
    \begin{tabular} {S[table-format=2.1]  S[table-format=2.0] S[table-format=2.1] S[table-format=3.0]  S[table-format=2.1]  
                     S[table-format=3.1]  S[table-format=2.1] S[table-format=3.0]}
        \toprule
        {$\theta \mathbin{/} \si{\degree}$} & {$N$} & {$\theta \mathbin{/} \si{\degree}$} & {$N$} 
    &   {$\theta \mathbin{/} \si{\degree}$} & {$N$} & {$\theta \mathbin{/} \si{\degree}$} & {$N$}\\
    \midrule
    26.0 & 56 & 27.0 & 105 & 28.0 & 212 & 29.0 & 138 \\
    26.1 & 58 & 27.1 & 119 & 28.1 & 215 & 29.1 & 125 \\
    26.2 & 54 & 27.2 & 125 & 28.2 & 218 & 29.2 & 111 \\
    26.3 & 62 & 27.3 & 141 & 28.3 & 215 & 29.3 & 107 \\
    26.4 & 58 & 27.4 & 154 & 28.4 & 208 & 29.4 & 95  \\
    26.5 & 68 & 27.5 & 157 & 28.5 & 189 & 29.5 & 77  \\
    26.6 & 72 & 27.6 & 166 & 28.6 & 189 & 29.6 & 73  \\
    26.7 & 83 & 27.7 & 180 & 28.7 & 176 & 29.7 & 58  \\
    26.8 & 89 & 27.8 & 188 & 28.8 & 164 & 29.8 & 56  \\
    26.9 & 95 & 27.9 & 211 & 28.9 & 149 & 29.9 & 53  \\
    {-}  & {-}& {-}  & {-} & {-}  & {-} & 30.0 & 53  \\
    \bottomrule
    \end{tabular}
\end{table}
\begin{figure}
    \centering
    \caption{Gemessene Winkel und Impulse zur Überprüfung der Bragg-Bedingung}
    \label{fig:Bragg}
    \includegraphics{build/Bragg.pdf}
\end{figure}
Das Maximum der Impulse $N_\text{max} = \num{218}$ ist bei einem Winkel von $\theta_\text{max} = \ang{28.2;;}$ gemessen worden.
Dieser Winkel weicht vom Sollwinkel $\theta_\text{soll} = \ang{28;;}$ nur um $\ang{0.2;;}$ ab, so dass die Bragg-Bedingung erfüllt ist.
\FloatBarrier
\subsection{Analyse eines Emissionsspektrums der Kupfer-Röntgenröhre} \label{sec:kupferlinien}
\begin{figure}
    \centering
    \caption{Das Emissionsspektrum der Kupfer-Röntgenröhre}
    \label{fig:Kupfer}
    \includegraphics{build/Kupfer.pdf}
\end{figure}
Die $K_\alpha$ Linie befindet sich bei $\theta = \ang{22.5;;}$, während die $K_\beta$ bei  $\theta = \ang{20.2;;}$ wiederzufinden ist.
Mittels linearer Interpolation lassen sich die Halbwertsbreiten bestimmen.
Dazu werden die beiden umliegenden Punkte, welche am nächsten an dem halben Maximum dran sind, verwendet um eine Interpolationsgerade
\begin{equation*}
    y = mx+b 
\end{equation*}
zu erhalten.
Aus dieser Geraden lässt sich der Winkel, bei der die Hälfte der maximalen Impulse gemessen wurde, bestimmen, wonach die Halbwertsbreite
aus der Differenz des Winkels links und rechts neben der $K_\alpha$ oder $K_\beta$ Linie ermittelt werden kann.
Der maximale Impuls liegt bei der $K_\beta$ Linie bei $1599$ Impulsen. 
Bei der $K_\alpha$ Linie wurde ein maximaler Impuls von $5050$ gemessen.
Die Interpolationsgeraden werden durch 
\begin{align*}
    y_{\beta, 1} &= 8359.99x - 166908.99    \\
    y_{\beta, 2} &= -8420.00x - 173877.00   \\
    y_{\alpha, 1} &= 35919.99x - -800479.99  \\
    y_{\alpha, 2} &= -31960.00x - 732785.00
\end{align*}
beschrieben.
Die Winkel der Halbwertsbreiten liegen bei 
\begin{align*}
    \theta_{\beta, 1} &= \ang{20.06;;}\\
    \theta_{\beta, 2} &= \ang{20.56;;}\\
    \theta_{\alpha, 1} &= \ang{22.36;;}\\
    \theta_{\alpha, 2} &= \ang{22.85;;} \; \text{,}
\end{align*}
wobei die Geraden und die eben genannten Winkel chronologisch zu der Abbildung \ref{fig:Kupfer} angeordnet  sind, d.h., dass in der Abbildung 
oben nach unten links nach rechts entspräche.
Die Interpolationsgeraden sind in der Abbildung \ref{fig:Kupfer} schwarz eingezeichnet.
Die Halbwertsbreiten betragen
\begin{align*}
    \symup{\Delta}K_\alpha  &= \ang{0.494;;} \\
    \symup{\Delta}K_\beta   &= \ang{0.495;;} \; \text{.}
\end{align*}
Mit Hilfe der Bragg Bedingung \eqref{eqn:bragg} lässt sich die Wellenlänge bestimmen, wonach die Energie bestimmt werden kann.
Die Energien ergeben sich zu 
\begin{align*}
    E_{K_\alpha} & =  \SI{8.04}{\kilo\electronvolt} \\
    E_{K_\beta}  & =  \SI{8.92}{\kilo\electronvolt} \\
    \symup{\Delta}E_{K_\alpha} &= \SI{165.84}{\electronvolt} \\
    \symup{\Delta}E_{K_\beta}  &= \SI{206.96}{\electronvolt} \; \text{.}
\end{align*}
Das Auflösungsvermögen 
\begin{equation}
    A_K = \frac{E_K}{\symup{\Delta}E_K}
\end{equation}
kann daraus zu 
\begin{align*}
    A_{K_\alpha}&= \num{48.50} \\
    A_{K_\beta} &= \num{43.08}
\end{align*}
bestimmt werden.
Mit Hilfe der Gleichungen \eqref{eqn:si1},\eqref{eqn:si2} und \eqref{eqn:si3} lassen sich die Abschirmkonstanten $\sigma_1$, $\sigma_2$ und $\sigma_3$ bestimmen.
Mit $E_\text{abs} = \SI{8987.96(15)}{\electronvolt}$\cite{eabs}, der Rydberg Energie $R_\infty$ und der Kernladungszahl
$z_\text{Kupfer} = 29 $ lässt sich
\begin{align*}
    \sigma_1 &= \num{3.292(21)}  \\
    \sigma_2 &= \num{12.30(13)}  \\
    \sigma_3 &= \num{22.29(74)}
\end{align*}
errechnen.
Der Fehler der n-ten Abschirmkonstante wird Mit Hilfe der Gauß-Fehlerfortpflanzung durch
\begin{equation}
    \symup{\Delta} \sigma_n = \left | \frac{\partial \sigma_n}{\partial E_\text{abs} } \symup{\Delta} E_\text{abs} \right | 
\end{equation} 
beschrieben.
\subsection{Bestimmung der Absorptionsenergien und -konstanten von verschiedenen Materialien}
Bei der Aufnahme der Absorptionsspektren wurde ein LiF-Kristall, eine Beschleunigungsspannung von $\SI{35}{\kilo\volt}$ und ein Strom von $\SI{1}{\milli\ampere}$
verwendet. 
In den jeweiligen Unterabschnitten sind die Intensitäten $\sfrac{\text{Imp}}{\si{\second}}$ gegen den Winkel $\theta$ aufgetragen. \\
Um die Winkel zu bestimmen, muss zunächst die Mitte der Kante bestimmt werden, welche sich mit Hilfe von
\begin{equation}
    I_K = I_{K, \text{min}} + \frac{I_{K, \text{min}} + I_{K, \text{max}}}{2}
\end{equation}
berechnen lässt. 
Folglich werden die beiden umliegenden Punkte betrachtet. 
Mit diesen wird eine Interpolation durchgeführt. 
Mit Hilfe der Interpolationsgerade $y = mx+b$ kann ermittelt werden, wo sich die Mitte der Kante befindet, wonach auf den zugehörigen Winkel 
geschlossen werden kann.
Die Interpolationsgeraden sind in den jeweiligen Abbildungen schwarz eingezeichnet. 
Die Energien lassen sich mit der Bragg-Bedingung \eqref{eqn:bragg} bestimmen, woraus sich nach Gleichung \eqref{eqn:abschirmkonstante} 
die Abschirmkonstante errechnen lässt.
\subsubsection{Zink}
Die Mitte der Kante liegt bei 
\begin{equation*}
    I_K = 78
\end{equation*}
, wobei die Interpolationsgerade durch 
\begin{equation*}
    y = 189.99 x - 3468.99
\end{equation*}
beschrieben wird.
Bei der Durchführung mit Zink liegt die Mitte der Absorptionskante ungefähr bei $\theta_\text{Zn} \approx \ang{18.67;;}$. Daraus ergibt sich eine Absorptionsenergie bzw. Abschirmkonstante von 
\begin{figure}
    \centering
    \caption{Absorptionsspektrum von Zink}
    \label{fig:zink}
    \includegraphics{build/zink.pdf}
\end{figure}
\begin{align*}
    E_{K, \text{abs}_\text{ Zn}}  &= \SI{9.617}{\kilo\electronvolt} \\
    \sigma_{K_\text{Zn}}         &= \num{3.611} \; \text{.}
\end{align*}
\FloatBarrier
\subsubsection{Gallium}
Die Mitte der Kante liegt bei 
\begin{equation*}
    I_K = 140.00
\end{equation*}
, wobei die Interpolationsgerade durch 
\begin{equation*}
    y = 17.34x -2334
\end{equation*}
beschrieben wird.
Die Mitte der Absorptionskante liegt ungefähr bei $\theta_\text{Ga} \approx \ang{17.34;;}$. Die daraus errechnete Energie bzw. Abschirmkonstante beträgt
\begin{figure}
    \centering
    \caption{Absorptionsspektrum von Gallium}
    \label{fig:Gallium}
    \includegraphics{build/gallium.pdf}
\end{figure}
\begin{align*}
    E_{K, \text{abs}_\text{ Ga}}  &= \SI{10.327}{\kilo\electronvolt} \\
    \sigma_{K_\text{Ga}}                &= \num{3.668} \; \text{.}
\end{align*}
\FloatBarrier
\subsubsection{Brom}
Die Mitte der Kante liegt bei 
\begin{equation*}
    I_K = 50.00
\end{equation*}
, wobei die Interpolationsgerade durch 
\begin{equation*}
    y = 13.19 x -642.00
\end{equation*}
beschrieben wird.
Bei Brom liegt die Kante ungefähr bei $\theta_\text{Br} \approx \ang{13.20;;}$. Somit lässt sich eine Absorptionsenergie bzw. Abschirmkonstante von 
\begin{figure}
    \centering
    \caption{Absorptionsspektrum von Brom}
    \label{fig:Brom}
    \includegraphics{build/brom.pdf}
\end{figure}
\begin{align*}
    E_{K, \text{abs}_\text{ Br}}  &= \SI{13.480}{\kilo\electronvolt} \\
    \sigma_{K_\text{Br}}                &= \num{3.835}
\end{align*}
errechnen.
\FloatBarrier
\subsubsection{Rubidium}
Die Mitte der Kante liegt bei 
\begin{equation*}
    I_K = 70.00
\end{equation*}
, wobei die Interpolationsgerade durch 
\begin{equation*}
    y = 11.77x - 787.00
\end{equation*}
beschrieben wird.
Die Mitte der Absorptionskante von Rubidium liegt ungefähr bei $\theta_\text{Rb} \approx \ang{11.77;;}$. Daraus ergibt sich eine Absorptionsenergie bzw. Abschirmkonstante von 
\begin{figure}
    \centering
    \caption{Absorptionsspektrum von Rubidium}
    \label{fig:Rubidium}
    \includegraphics{build/rubidium.pdf}
\end{figure}
\begin{align*}
    E_{K, \text{abs}_\text{ Rb}}  &= \SI{15.089}{\kilo\electronvolt} \\
    \sigma_{K_\text{Rb}}         &= \num{4.067} \; \text{.}
\end{align*}
\FloatBarrier
\subsubsection{Strontium}
Die Mitte der Kante liegt bei 
\begin{equation*}
    I_K = 310.00
\end{equation*}
, wobei die Interpolationsgerade durch 
\begin{equation*}
    y = 11.09x -3321.00
\end{equation*}
beschrieben wird.
Die in der Abbildung \ref{fig:Strontium} hervorgehende Mitte der Absorptionskante liegt ungefähr bei $\theta_\text{Sr} \approx \ang{11.09;;}$. 
Daraus ergibt sich eine Absorptionsenergie bzw. Abschirmkonstante von 
\begin{figure}
    \centering
    \caption{Absorptionsspektrum von Strontium}
    \label{fig:Strontium}
    \includegraphics{build/strontium.pdf}
\end{figure}
\begin{align*}
    E_{K, \text{abs}_\text{ Sr}}  &= \SI{15.998}{\kilo\electronvolt} \\
    \sigma_{K_\text{Sr}}         &= \num{4.109} \; \text{.}
\end{align*}
\FloatBarrier
\subsubsection{Zirkonium}
Die Mitte der Kante liegt bei 
\begin{equation*}
    I_K =  449.99
\end{equation*}
, wobei die Interpolationsgerade durch 
\begin{equation*}
    y = 9.96x -4274.99
\end{equation*}
beschrieben wird.
Bei Zirkonium liegt die Mitte der Absorptionskante liegt ungefähr bei $\theta_\text{Zr} \approx \ang{9.96;;}$. 
Daraus ergibt sich eine Absorptionsenergie bzw. Abschirmkonstante von 
\begin{figure}
    \centering
    \caption{Absorptionsspektrum von Zirkonium}
    \label{fig:Zirkonium}
    \includegraphics{build/zirkonium.pdf}
\end{figure}
\begin{align*}
    E_{K, \text{abs}_\text{ Zr}}  &= \SI{17.800}{\kilo\electronvolt} \\
    \sigma_{K_\text{Zr}}         &= \num{4.296} \; \text{.}
\end{align*}
\FloatBarrier
\subsection{Berechnung der Rydbergenergie}
Um die Rydbergfrequenz zu bestimmen kann das Moysley'sche Gesetz mit n = 1 zu 
\begin{equation}
    \sqrt{E_K} = \sqrt{Rh}z - \sqrt{Rh}\sigma_k
\end{equation}
umgestellt werden.
In Regressionsparameter überführt nimmt die Beziehung die Gestalt
\begin{equation}
    y = mz+b
\end{equation}
an.
Mittels Durchführung der linearen Regression lassen sich die Regressionsparameter zu 
\begin{align*}
        m & = \left( \num{3.538(17)} \right)   \sqrt{\si{\electronvolt}} \\
        b & = \left( \num{-8.024(636)} \right) \sqrt{\si{\electronvolt}}
\end{align*}
\begin{figure}
    \centering
    \caption{Lineare Regresssion zur Bestimmung der Rydbergfrequenz}
    \label{fig:rydberg}
    \includegraphics{build/rydberg.pdf}
\end{figure}
bestimmen.
Daraus kann die Rydbergfrequenz mit
\begin{equation*}
    R = \frac{m^2}{h}
\end{equation*}
zu 
\begin{equation*}
    R = \SI{3.028(31)e15}{\hertz}
\end{equation*}
errechnet werden.
Abschließend lässt sich die Rydbergenergie aus der Formel
\begin{equation}
    R_\infty = \symup{h}R
\end{equation}
zu
\begin{equation*}
    R_\infty = \SI{12.52(13)}{\electronvolt}
\end{equation*}
bestimmen.
Die Unsicherheiten der Rydbergfrequenz und der Rydbergenergie ist
\begin{align}
    \symup{\Delta} R        & = \left | \frac{2m}{\symup{h}} \symup{\Delta}m \right | \\
    \symup{\Delta} R_\infty & = \symup{h} \symup{\Delta} R \; \text{.}
\end{align}