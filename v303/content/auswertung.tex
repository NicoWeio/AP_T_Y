\section{Auswertung}
\label{sec:Auswertung}Jegliche Fehlerrechnung wurde mit der python-Bibliothek uncertainties \cite{uncertainties} absolviert.
Trotz dessen sind die Formeln für die Unsicherheiten in den jeweiligen Abschnitten angegeben.
Allgemeine Rechnungen wurden mit der python-Bibliothek numpy \cite{numpy} automatisiert. 
Die graphischen Unterstützungen wurden mit Hilfe der python-Bibliothek matplotlib \cite{matplotlib}
\subsection{Verifizierung der Funktion eines phasenempfindlichen Gleichrichters}
In der Tabelle \ref{tab:voltage} sind die gemessenen Ausgangsspannungen in Abhängigkit von der Phase
$\Phi$ aufgezeichnet. 
Dabei ist $U_\text{out}$ die unverrauschte Spannung und $U_\text{out}$ die verrauschte Spannung.
\begin{table}
    \centering
    \caption{Gemessen Ausgangsspannungen $U_\text{out}$ und $U_\text{out, noise}$}
    \label{tab:voltage}
    \begin{tabular} {S[table-format=3.0] S[table-format=3.0] S[table-format=3.0]}
        \toprule
        {$\Phi \mathbin{/} \si{\degree}$} & {$U_\text{out} \mathbin{/} \si{\milli\volt}$} & {$U_\text{out, noise} \mathbin{/} \si{\milli\volt}$}\\
    \midrule
    0       & 96 & 96\\
    30      & 86 & 86\\
    60      & 58 & 58\\
    90      & 74 & 74\\
    120     & 110 & 110\\
    150     & 100 & 100\\
    180     & 96 & 96\\    
    \bottomrule
\end{tabular}
\end{table}
Aus der Tabelle \ref{tab:voltage} ergibt sich einerseits die Abbildung \ref{fig:Uwo} mit den unverrauschten 
Spannungen, anderseits die Tabelle \ref{fig:Uwi} mit den verrauschten Sapnnungen.
Dort sin die gemessenen Spannungen gegen die Phase aufgetragen.
Zusätzlich ist dort esine Ausgleichskurve eingezeichnete welche sich mittels 
\begin{equation}
    U_\text{out} = a\cos \left ( b \, \Phi + c \right ) + d
\end{equation}
beschreiben lässt, wobei der Parameter $a$ die Amplitude $U_0$ ist.
Für die unverrauschte Spannung ergeben sich die Parameter zu    
\begin{align*}
    a &= \num{22.09(594)}       \\
    b &= \num{2.17(22)}         \\
    c &= \num{113.93(44)}       \\
    d &= \num{86.58(413)} \; \text{.}
\end{align*}
Die Regressionsparameter der verrauschten Messwerte werden zu
\begin{align*}
    a_\text{noise} &= \num{-0.63(32)} \\           
    b_\text{noise} &= \num{2.81(40)}  \\            
    c_\text{noise} &= \num{112.03(79)}\\
    d_\text{noise} &= \num{3.30(22)}
\end{align*}
berechnet.
\begin{figure}
    \centering
    \caption{Gemessene Spannungen $U_\text{out}$ mit Regressionskurve}
    \label{fig:Uwo}
    \includegraphics{build/Uwo.pdf}
\end{figure}
\begin{figure}
    \centering
    \caption{Gemessene Spannungen $U_\text{out, noise}$ mit Regressionskurve}
    \label{fig:Uwi}
    \includegraphics{build/Uwi.pdf}
\end{figure}
Somit können die Spannungsaplituden $U_0$ zu 
\begin{align*}
    U_0                 & = \SI{22.09(594)}{\milli\volt} \\
    U_{0 \text{, noise}}& = \SI{-0.63(32)}{\milli\volt}
\end{align*}
bestimmt werden.
\subsection{Überprüfung der Rauschunterdückung des Lock-In-Verstärkers mit einer Photodetektorschaltung}
\begin{table}
    \centering
    \caption{Gemessen Spannungen $U$ und in Abhängigkeit des Abstands}
    \label{tab:distance}
    \begin{tabular} {S[table-format=2.1] S[table-format=1.2]}
        \toprule
        {$r \mathbin{/} \si{\centi\metre}$} & {$U \mathbin{/} \si{\volt}$}\\
    \midrule
    4.7  &1.90 \\
    5.0  &1.70 \\
    6.0  &1.50 \\
    7.0  &1.40 \\
    8.0  &1.30 \\
    9.0  &1.15 \\
    10.0 & 1.00\\
    11.0 & 0.85\\
    12.0 & 0.70\\
    13.0 & 0.60\\
    14.0 & 0.50\\
    15.0 & 0.45\\
    16.0 & 0.40\\
    17.0 & 0.35\\
    18.0 & 0.30\\
    19.0 & 0.30\\
    20.0 & 0.25\\
    21.0 & 0.23\\
    28.0 & 0.10\\
    50.0 & 0.05\\
    80.0 & 0.04\\  
    \bottomrule
\end{tabular}
\end{table}
\begin{figure}
    \caption{Gemessene Spannungen und Regressionskurven in Abhängigkeit des Abstands}
    \label{fig:distance}
    \includegraphics{build/distance.pdf}
\end{figure}