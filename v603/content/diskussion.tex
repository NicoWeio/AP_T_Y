\section{Diskussion}
\label{sec:Diskussion}
Die in dem Abschnitt \ref{sec:energy} berechneten Werte für die Energien für die $K_\alpha$ und $K_\beta$ Linien
$E_{K_\alpha} = \SI{8.04}{\electronvolt}$ und $E_{K_\beta} = \SI{8.92}{\electronvolt}$ weichen von den Literaturwerten
$E_{K_\alpha, \text{ Lit}} = \SI{8.048}{\electronvolt}$\cite{litlinien} und $E_{K_\alpha, \text{ Lit}} = \SI{8.905}{\electronvolt}$\cite{litlinien} um
\begin{align*}
    \eta_\alpha &= \SI{0.1}{\percent} \\
    \eta_\beta  &= \SI{0.17}{\percent}
\end{align*}
ab. 
Somit lässt sich sagen, dass die Energien mit einer hohen Genauigkeit bestimmt werden konnte.\\
Die Compton-Wellenlänge des Elektrons wurde zu $\lambda_\text{c} = \SI{3.76(6)}{\pico\metre}$ bestimmt. 
Diese weicht von dem Literaturwert $\lambda_\text{c, Lit} = \SI{2.43}{\pico\metre}$\cite{compton} um 
\begin{equation*}
    \eta_{\lambda_\text{c}} = \SI{54.73}{\percent}
\end{equation*}
ab.  
Die relativ hohe Abweichung könnte durch die Auslassung der Totzeitkorrektur\eqref{eqn:correction} zu Stande gekommen sein.
Anderseits könnte die Abweichung auch durch die lange Integrationszeit enstanden sein, da sich während der Messung der Impulse 
die Fehler aufsummieren könnten.