\section{Durchführung}
\label{sec:Durchführung}
\subsection{Vorbereitung}
Als Vorbereitungsaufgabe sollte man sich über die Dichte $\rho$, die spezifische Wärme $c$ und die Wärmeleitfähigkeit $\kappa$ für 
Aluminium, Messing, Edelstahl und Wasser. Da die Werte von Wasser nicht relevant sind, werden dies nicht aufgeführt.
\begin{table}
    \centering
    \caption{Literaturwerte von Aluminium, Messing und Edelstahl}
    \label{tab:litwerte}
    \begin{tabular}{c c c c}
        \toprule
        $\text{Metall}$ & $\rho\,([\rho]=\si{\kilo\gram\meter\tothe{-3}})$ 
        & $c\, ([c]=\si{\joule\kilo\gram\kelvin\tothe{-1}})$ & $\kappa\,([\kappa]=\si{\watt\meter\tothe{-1}\kelvin\tothe{-1}})$\\
        \midrule
        Aluminium & 2800 & 830& 236 \\
        Messing & 8520 & 385 & 120\\
        Edelstahl &8000 & 400 & 21\\
        \bottomrule
    \end{tabular}
\end{table}
\subsection{Aufbau}
Das Experiment wird wie in Abbildung \ref{fig:aufbau} aufgebaut. Auf der Grundplatte befinden sich vier rechteckige Probestäbe aus den Materialien:
Aluminium, Messing(2x) und Edelstahl. Ein Peltierelement heizt bzw. kühlt die vier Probestäbe simultan bei einer Spannung von $U_P=\SI{5}{\volt}$
und $U_P=\SI{8}{\volt}$ bei der dynamischen Methode. An jedem Probestab wird an zwei Stellen die Temperatur gemessen und über ein 'Temperatur Array'
an einen Datenlogger (Xplorer GLX) weitergegeben. Die Daten können über den Datenlogger simultan gemessen und aufgezeichnet werden.
\begin{figure}
    \centering
    \includegraphics[scale=0.7]{Aufbau.pdf}
    \caption{Aufbau des Experiments}
    \label{fig:aufbau}
\end{figure}
\subsection{Versuchsdurchführung}
\label{sec:Vdurch}
