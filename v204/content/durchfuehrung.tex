\section{Durchführung}
\label{sec:Durchführung}
\subsection{Vorbereitung}
Als Vorbereitungsaufgabe sollte man sich über die Dichte $\rho$, die spezifische Wärme $c$ und die Wärmeleitfähigkeit $\kappa$ für 
Aluminium, Messing, Edelstahl und Wasser. Da die Werte von Wasser nicht relevant sind, werden dies nicht aufgeführt.
\begin{table}
    \centering
    \caption{Literaturwerte von Aluminium, Messing und Edelstahl}
    \label{tab:litwerte}
    \begin{tabular}{c c c c}
        \toprule
        $\text{Metall}$ & $\rho \mathbin{/} \si{\kilogram\meter\tothe{-3}}$ 
        & $c \mathbin{/} \si{\joule\kilogram\tothe{-1}\kelvin\tothe{-1}}$ & $\kappa \mathbin{/} \si{\watt\meter\tothe{-1}\kelvin\tothe{-1}}$\\
        \midrule
        Aluminium & 2800 & 830 & 236 \\
        Messing   & 8520 & 385 & 120 \\
        Edelstahl & 8000 & 400 & 21  \\
        \bottomrule
    \end{tabular}
\end{table}
\subsection{Aufbau}
Das Experiment wird wie in Abbildung \ref{fig:aufbau} aufgebaut. Auf der Grundplatte befinden sich vier rechteckige Probestäbe aus den Materialien:
Aluminium, Messing(2x) und Edelstahl. Ein Peltierelement heizt bzw. kühlt die vier Probestäbe simultan bei einer Spannung von $U_P=\SI{5}{\volt}$
und $U_P=\SI{8}{\volt}$ bei der dynamischen Methode. An jedem Probestab wird an zwei Stellen die Temperatur gemessen und über ein 'Temperatur Array'
an einen Datenlogger (Xplorer GLX) weitergegeben. Die Daten können über den Datenlogger simultan gemessen und aufgezeichnet werden. 
Es wird außerdem eine Stoppuhr für das dynamische Verfahren benötigt.
\begin{figure}
    \centering
    \includegraphics[scale=0.7]{Aufbau.pdf}
    \caption{Aufbau des Experiments}
    \label{fig:aufbau}
\end{figure}
\subsection{Versuchsdurchführung}
\label{sec:Vdurch}
Zuerst muss die Verkabelung überprüft werden und gegebenenfalls korrigiert werden. 
Zudem wird der Abstand x zwischen den Thermoelementen gemessen.
Um den Wärmeaustausch mit der Umgebung zu minimieren, wird bei jeder
Messung die Wärmeisolierung über die Stäbe gelegt. Nach jeder Messung müssen die Isolierungen entfernt werden und die Stäbe werden mit Hilfe des Peltier Elements gekühlt.
Das System ist träge und benötigt Zeit um abzukühlen. Die gesammelten Daten werden nach der Messung auf einen USB-Stick gezogen und können dann benutzt werden.\\
\subsubsection{Statische Methode}
Es wird an jedem Stab an jeweils zwei Stellen die Temperatur als Funktion der Zeit gemessen. 
Der Datenlogger wird auf eine Abtastzeit von $\symup{\Delta} t_{GLX}=\SI{5}{\second}$ eingestellt, während die Spannung für das Peltier Element auf $U_P=5\si{\volt}$ gestellt wird.
Die Messung wird durchgeführt bis das Thermoelement T7 die Temperatur $\SI{45}{\celsius}$ erreicht. Danach wird die Messung beendet und die Probestäbe werden gekühlt.
Es werden die Temperaturverläufe für T1 und T4, sowie für T5 und T8 in jeweils einer Graphik dargestellt.
\subsubsection{Dynamische Methode}
Es wird das Angström-Meßverfahren verwendet, bei dem die Probestäbe periodisch geheizt und gekühlt werden.
Der Datenlogger wird auf eine Abtastzeit von $\symup{\Delta}t_{GLX}=\SI{2}{\second}$ eingestellt.
Bevor die neue Messung erfolgt, sollte die Temperatur der Thermoelemente unter $\SI{30}{\celsius}$ sein.
Die erste dynamische Messung hat eine Periode von $\SI{80}{\second}$, dabei werden die Probestäbe $\SI{40}{\second}$ geheizt und dann $\SI{40}{\second}$ gekühlt, ohne die Isolierung zu entfernen.
Die Messung wird beendet nach mindestens 10 Perioden und anschließend wird wieder gekühlt.
Nachdem die Themoelemente wieder hinreichend gekühlt sind, wird der Versuch mit einer Periodendauer von $\SI{200}{\second}$ wiederholt.
Die Messung wird gestoppt, sobald eins der Thermoelemente über $\SI{80}{\celsius}$ fällt.
Zuletzt wird noch einmal gekühlt und dann ausgewertet.