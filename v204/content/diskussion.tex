\section{Diskussion}
\label{sec:Diskussion}
Bei Betrachtung der Graphen von Abbildung \ref{stat2}, fällt auf das die Graphen von T1, T4 und T5, jeweils eine Anomalie bei $t \approx \SI{700}{\second}$ haben.
Jedoch ist bei dem Graphen von T8 so etwas nicht zu erkennen. Da alle Messungen gleichzeitig und mit gleichem Aufbau ausgeführt wurden, 
könnte dies eine Materialabhängigkeit sein. Da aber der Graph von T8 anders verläuft als der Rest, ist die Abweichung vielleicht zu klein um sie zum Erkennen.
Wenn T8 auch eine Anomalie hätte, könnte man auf eine Schwankung des Peltier-Elements oder des Netzgeräts schließen.
Die Fehler ließen sich untersuchen, wenn man die Geräte austauscht und die Daten erneut auswerten würde. 
Zudem wäre für mehr Präzision, eine mehrfache Wiederholung des Versuches notwendig. Um Fehler mit Hilfe von Mittelwerten zu dezimieren.
\\
Bei der dynamischen Methode liegen relativ große Abweichungen von den Literaturwerten \ref{tab:litwerte} vor:
\begin{align*}
    \kappa_\text{Messing, gemessen} &= \SI{81 \pm 9}{\watt\per\metre\per\kelvin} & \kappa_\text{Messing, Literatur} &= \SI{120}{\watt\per\metre\per\kelvin}\\
    \kappa_\text{Aluminium, gemessen} &= \SI{168 \pm 8}{\watt\per\metre\per\kelvin} & \kappa_\text{Aluminium, Literatur} &= \SI{236}{\watt\per\metre\per\kelvin}\\
    \kappa_\text{Edelstahl, gemessen} &= \SI{6 \pm 1.7}{\watt\per\metre\per\kelvin} & \kappa_\text{Edelstahl, Literatur} &= \SI{21}{\watt\per\metre\per\kelvin}
\end{align*}
Die relativen Abweichungen betragen:
\begin{align*}
    \implies \symup{Abweichung}_{\symup{Messing}} &\hat{=} 32.5 \%\\
    \implies \symup{Abweichung}_{\symup{Aluminium}} &\hat{=} 28.9\%\\
    \implies \symup{Abweichung}_{\symup{Edelstahl}} &\hat{=} 70.15 \%
\end{align*}
Auffällig ist, dass alle berechneten Wärmeleitfähigkeiten deutlich unter den Literaturwerten liegen.
Eine genauere Messung wäre möglich, wenn man mehr Messpunkte auf einen Probestab plaziert. So könnte man die Wegabhängigkeit besser ermitteln.
Durch das manuelle umschalten des Peltier-Elements, kommt es zu Abweichungen in der Periode und Frequenz, was auch Einfluss auf die weitere Auswertung hat.
\\
Messfehler entsprangen auch den nicht komplett Isolierten Probestäben, so dass diese mit der Umgebung Interagieren konnten und Wärme abgaben.
Da die Aperaturen, welche wir benutzt haben, schon relativ alt waren, könnte auch dies zu einigen Störungen geführt haben.
\\
Interessant ist auch der große Relativfehler von dem Edelstahlwert. Dieser könnte nur durch schlechte Isolierung oder durch einen bereits beschädigten Probestab geschehen sein.


