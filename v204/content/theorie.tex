\section{Theorie}
\label{sec:Theorie}
Die Inhalte des Theorieteils, sind auf Grundlage von \cite{sample} und \cite{demtröder}, formuliert.
Ein Wärmetransport findet statt, wenn Temperaturdifferenzen zwischen zwei Orten auftreten.
Dieser Wärmetransport geht entlang des Temperaturgefälles. Grundsätzlich gibt es drei verschiedene Mechanismen des Wärmetransportes.
Dabei unterscheidet man zwischen: der Konvektion, der Wärmeleitung und der Wärmestrahlung.
In diesem Versuch behandeln wir, jedoch nur die Wärmeleitung von Metallen.
Der Wärmetransport mittels Wärmeleitung erfolgt über Phononen und frei bewegliche Elektronen. Der Betrag des Gitters ist bei Metallen zu vernachlässigen.
\subsection{Statische Methode}
\label{sec:stM}
Betrachtet wird ein Stab der Länge L und der Querschnittsfläche A mit der Dichte $\rho$ und der spezifischen Wärmekapazität $c$ des Materials.
Wird ein Ende des Stabes erhitzt, so fließt in der Zeit $dt$ durch die Querschnittsfläche A die Wärmemenge
\begin{equation}
    \frac{dQ}{dt}=-\kappa A \frac{\partial T}{\partial x} \,.
    \label{eqn:Wärmemenge}
\end{equation}
Dabei ist $\kappa$ ([$\kappa$]=$\frac{\symup{W}}{\symup{m K}}$) die Wärmeleitfähigkeit des Materials. Da der Wärmestrom 
in Richtung des kälteren Ortes verläuft, ist das Minuszeichen in der Gleichung Konvention. Die Wärmestromdichte $j_w$ lässt sich berechnen durch
\begin{equation}
    j_w=-\kappa \frac{\partial T}{\partial x}\,.
    \label{eqn:wärmedichte}
\end{equation}
Aus dieser Gleichung und der Kontinuitätsgleichung lässt sich die eindimensionale Wärmeleitungsgleichung
\begin{equation}
    \frac{\partial T}{\partial t}=\frac{\kappa}{\rho c}\frac{\partial^2 T}{\partial x^2}
    \label{eqn:Wärmeleitungsgleichung}
\end{equation}
herleiten. Die Wärmeleitungsgleichung beschreibt die räumliche und zeitliche Entwicklung der Temperaturverteilung. Dabei ist die Größe $\sigma_T=\sfrac{\kappa}{\rho c}$
die Temperaturleitfähigkeit. Die Temperaturleitfähigkeit ist eine Materialeigenschaft, die die 'Schnelligkeit' angibt, mit der sich ein Temperaturunterschied ausgleicht.
Die Lösung der Wärmeleitungsgleichung ist abhängig von den Anfangsbedingungen und der Stabgeometrie.
\subsection{Dynamische Methode}
\label{sec:dyM}
Wird ein Stab mit der Periode $\tau$ erwärmt und abgekühlt, so entsteht eine räumliche und zeitliche Temperaturwelle
\begin{equation}
    T(x,t)=T_\text{max} \exp \left({-\sqrt{\frac{w \rho c}{2 \kappa}}}\, x \right)\cos \left( wt-\sqrt{\frac{w \rho c}{2 \kappa}}\, x\right)\; \text{,}
\end{equation}
welche sich über den Stab fortsetzt. Die Phasengeschwindigkeit $v$ der Welle ist folglich
\begin{equation}
    v=\frac{w}{k}=\frac{w}{\sqrt{\frac{w \rho c}{2 \kappa}}}=\sqrt{\frac{2\kappa w}{\rho c}}\; \text{.}
\end{equation}
Die Wellenlänge lässt sich jetzt berechnen aus
\begin{equation}
    \lambda= \frac{v/f}= \frac{\sqrt{\frac{2\kappa w}{\rho c}}}{f} \, \text{.}
    \label{eqn:wellenlänge}
\end{equation}
Dabei ist $f$ die Frequenz der Welle.
Das Amplitudenverhältnis $A_\text{nah}$ zu $A_\text{fern}$, an zwei Messstellen $x_\text{nah}$ und $x_\text{fern}$, ergibt die Dämpfung der Welle.
Mit $w=\frac{2\pi}{\tau}$ und $\phi=2\pi \frac{\symup{\Delta}t}{\tau}$\footnote{die Phase $\phi$ und die Periodendauer $\tau$} erhält man für die Wärmeleitfähigkeit
\begin{equation}
    \kappa=\frac{\rho\, c (\symup{\Delta}x)^2}{2\symup{\Delta}t \ln (A_\text{nah}/A_\text{fern})}\; \text{.}
\end{equation}
Dabei beschreibt $\symup{\Delta}x$ den Abstand der Messstellen und $\symup{\Delta}t$ die Phasendifferenz der Temperaturwelle zwischen den beiden Messstellen.

