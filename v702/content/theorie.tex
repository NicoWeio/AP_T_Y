\section{Theorie}
\label{sec:Theorie}
Das Ziel dieses Versuches ist es die Halbwertszeiten von verschiedenen Isotopen zu bestimmen.
\\
Stabile Atomkerne brauchen ein Verhältnis von Neutronen zu Protonen, welches in einem bestimmten Bereich liegt. Liegt dieses Verhältnis außerhalb eines Stabilitätsbereiches
wandelt sich der Kern mit sehr unterschiedlicher Wahrscheinlichkeit in einen stabilen oder einen anderen instabilen Kern, der dann weiter zerfällt, um.
Die Halbwertszeit $T$ ist der Zeitraum, in dem die Hälfte einer Anzahl von instabilen Kernen, zerfallen ist. Mit dem Versuch können Halbwertszeiten, in der Größenordnung
Sekunden bis Stunden, bestimmt werden. Es ist notwendig solche Nuklide unmittelbar vor Beginn der Messung herzustellen. Um ein Nuklid zu erzeugen 
ist es am einfachsten Stabile Kerne mit Neutronen zu beschießen.
\subsection{Kernreaktionen mit Neutronen}
Bei einer Kernreaktion, bei der ein Neutron in einen Kern $A$ eintritt, entsteht bei Absorption des Neutrons ein neuer Kern $A*$.
Dieser neue Kern wird Zwischenkern oder Compoundkern genannt. Dabei regt die Energie des einfallenden Neutrons den Zwischenkern an. Die Energie verteilt sich innerhalb des Kerns auf die Nukleonen.
Durch die Verteilung der Energie ist der Kern $A*$ meist nicht mehr in der Lage ein Nukleon abzustoßen. Durch Emission eines $\gamma$-Quants nach etwa $\SI{10e-16}{\second}$
geht der Zwischenkern wieder in seinen Grundzustand über. 
Diese Reaktion wird durch folgende Gleichung beschrieben
\begin{equation*}
    \ce{{^m_z A} + \ce{^1_0 n} -> {^{m+1}_z A^*} -> {^{m+1}_z A} } + \gamma \, .
\end{equation*}
Meistens sind die neu entstandenen Kerne $\ce{^{m+1}_z A}$ nicht stabil, jedoch Langlebiger aufgrund des Energieverlustes.
Durch Emission eines Elektrons und eines Antineutrinus wandelt er sich in einen stabilen Kern um:
\begin{equation*}
    \ce{^{m+1}_z A -> {^{m+1}_{z+1} C} + $\beta$- }+\symup{E_{Kin}} + \bar{\nu}_e
\end{equation*}
Der Wirkungsquerschnitt beschreibt die Wahrscheinlichkeit für das Einfangen eines Neutrons durch einen stabilen Kern. 
Es lässt sich zeigen, dass der Wirkungsquerschnitt antiproportional zu der Geschwindigkeit des Neutrons ist. Somit ist es bei langsameren Neutronen wahrscheinlicher das sie absorbiert werden.
Die Ursache dafür ist der längere Aufenthalt in der nähe des Atomkerns.
\subsection{Neutronen erzeugen}
Das Neutron ist als freies Teilchen instabil und kommt somit nicht in der Natur vor. Durch geeignete Kernreaktionen können Neutronen erzeugt werden.
In diesen Versuch werden die Neutronen durch beschuss von $\ce{^9 Be}$-Kernen mit $\alpha$-Teilchen freigesetzt:
\begin{equation*}
    \ce{ ^9_4 Be + ^4_2 $\alpha$ -> ^{12}_6 C + ^1_0 n} \, .
\end{equation*}
Aus dem Zerfall von $\ce{^226 Ra}$-Kernen kommen die $\alpha$-Teilchen für die Kernreaktion. 
Um die gewonnenen Neutronen zu bremsen werden sie durch dicke Materieschichten, aus Atomen mit leichten Kernen, geschickt.
Dabei geben die Neutronen ihre Energie durch elastische Stöße an die leichten Kerne ab, wobei die Energieabgabe mit der Differenz der stoßenden Massen abnimmt.
Damit eignen sich leichte Materie-Kerne am besten, da bei ähnlichen Massen die Energieabgabe am größten ist.
Proben die mit diesen Neutronen beschossen werden, wandeln sich unter $\beta$-Strahlung wieder in stabile Isotope um
\subsection{Der Zerfall instabiler Isotope}
Der Zerfall instabiler Isotope verläuft exponentiell. Die Anzahl der noch nicht zerfallenen Kerne kann mit 
\begin{equation}
    \symup{N(t)=N_0 e^{- \lambda t}}
    \label{eqn:isotope}
\end{equation}
berechnet werden. Das $\symup{N_0}$ ist hierbei die zum Zeitpunkt $t=0$ vorhandenen instabilen Kerne und $\symup{\lambda}$ die Zerfallskonstante, welche eine Aussage über die Wahrscheinlichkeit des Zerfalls macht.
Um die Halbwertszeit zu bekommen, wird für $\symup{N(t)}$ einfach $\frac{N_0}{2}$ eingesetzt. Durch Umformungen folgt
\begin{equation}
    \symup{T}=\frac{\ln{2}}{\lambda} \, .
    \label{eqn:halbw}
\end{equation}
Da die Bestimmung der zerfallenen Kerne einfacher ist, als die Bestimmung der übrigen Kerne, wird in einem Zeitintervall $\symup{\Delta}t$
die Anzahl der zerfallenen Kerne gemessen. Für die Anzahl der zerfallenen Kerne im Zeitraum $\symup{\Delta}t$ gilt
\begin{equation}
    \symup{N_{\Delta t}=N_0(1-e^{-\lambda \Delta t})e^{-\lambda t}} \, .
\end{equation}
Eine Schwierigkeit ist jedoch das Zeitintervall $\symup{\Delta t}$ richtig zu wählen, da
bei zu kleinem $\symup{\Delta t}$ große statistische Fehler für $\symup{N_{\Delta t}}(t)$ entstehen 
und bei zu großen $\symup{\Delta t}$ erhält man einen systematischen Fehler bei $\lambda$. 
Deshalb sind in dem Versuch die Zeitintervalle vorgegeben. 
Es gibt instabile Isotope bei denen nicht nur ein Zerfallsprodukt entsteht.
Wird natürliches Rhodium, das zu $100\%$ aus dem Isotop $\ce{^103 Rh}$ besteht, dann entsteht, wie in der Gleichung
\begin{equation}
   \ce{^103_45 Rh + n } 
    \begin{dcases}
        \stackrel{10\%}{\ce{->}}\ce{ ^104i_45 Rh &-> ^104_45 Rh + $\gamma$ -> ^104_46 Pd + $\beta$- + $\bar{\nu}_e$}  \\
        \stackrel{90\%}{\ce{->}}\ce{ ^104_45 Rh  &-> ^104_46 Pd + $\beta$- + $\bar{\nu}_e$} 
    \end{dcases}
\end{equation}
zu sehen, mit einer Wahrscheinlichkeit von $0,9$ das instabile Isotop $\ce{^104 Rh}$ sowie das instabile $\ce{^104i Rh}$, welches nach Emission eines $\gamma$-Quants in $\ce{^104 Rh}$ übergeht.
Die Zerfälle laufen simultan ab aber mit verschiedenen Halbwertszeiten. Da auch die $\gamma$-Strahlung detektiert wird, ist die Gesamtaktivität die Summe der beiden einzelnen Aktivitäten.
Trotzdem lassen sich die Halbwertszeiten der beiden Isotope gut bestimmen, da nach einer bestimmten Zeit $t^*$ nur noch das langlebigere Isotop für die Aktivität verantwortlich ist.
Aufgrund des kleinen Effekts werden, die bei dem Zerfall des $\ce{^104i_45 Rh}$ entstehenden und weiter zerfallenden Kerne des $\ce{^104_45 Rh}$, in der Gesamtzählrate vernachlässigt. 
Die Bestimmung der Zerfallskonstanten erfolgt dann nach einem einfachen Prinzip. Erst wird die Zerfallskonstante nach $t^*$ bestimmt und dann wird damit auf die Zerfallskonstante der kurzlebigen Isotope geschlossen.