\section{Diskussion}
\label{sec:Diskussion}
Nach der Berechnung der Verhältnisse der errechneten Halbwertszeiten und der literaturwerten
(\ce{ ^104_45 Rh} \cite{rh104}, \ce{ ^{104\text{i}}_45 Rh} \cite{rh104i}, \ce{ ^52_23} \cite{vana}) 
\begin{align*}
  \eta_\text{V}       &=   \frac{219}{244.6} = \SI{89.53}{\percent} \\
  \eta_\text{Rh, kurz} &=   \frac{41}{42.3}  = \SI{96.92}{\percent} \\
  \eta_\text{Rh, lang} &=   \frac{230}{260}  = \SI{88.46}{\percent} 
\end{align*}
wird auffällig, dass die errechnete Halbwertszeit von \ce{ ^104_45 Rh} nur um ca. $\SI{4}{\percent}$ abweicht.
Jedoch weicht der errechnete Wert von \ce{ ^{104\text{i}}_45 Rh} stärker {ca. $\SI{12}{\percent}$} von dem literaturwert ab.
Dort tritt die Vermutung auf, dass die Wahl von $t^*$ von dem optimalsten $t^*$ abweicht.
Jedoch beeinflusst der kurzlebige Zerfall den langlebigen Zerfall auch noch nach $t^*$+, so dass dieser (auch wenn nur geringer) Einflussfaktor
zu dieser Abweichung führt. Außerdem sticht das Verhältnis von Vanadium hervor.
Diese müsste im Vergleich zu Rhodium gering sein, da dort nur eine Ausgleichsrechung ohne Bestimmung von jeglichen Punkten 
durchgeführt werden musste. Dies könnte auf einen Fehler bei der Messung hindeuten.