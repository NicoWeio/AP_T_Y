\section{Diskussion}
\label{sec:Diskussion}
Der Stab 1 ist braun und quaderförmig, die Probe 2 braun und rund, der Stab 3 quaderförmig und silber und der vierte Stab braun/gelb und rund.
Bei Betrachtung der Tabelle \ref{tab:regression} fällt auf, dass der Staab 1 ein besonders großes Elastizitätsmodul aufweist, wobei die 
Elastizitätsmodule der anderen drei Stäbe in einem Bereich von ca. $90$ bis $\SI{230}{\giga\pascal}$ bleiben.
Ebenfalls werden hohe Abweichung in der Abbildung des ersten Stabs \ref{fig:probe1single} ersichtlich, da die Messwerte
einen beinahe oszillatorisches Verhalten aufweisen statt linear auf einer Geraden zu verlaufen.
Somit lässt sich sagen, dass der hohe Elastizitätsmodul durch eine Messungenauigkeit enstanden sein könnte.
Dies wird durch die Tatsache gestärkt, dass der Stab bereits vor der Versuchsdurchführung stark verbogen war, so dass dort
systematische Fehler aufkommen können.
Dieser hohe Elastizitätsmodul des ersten Stabs wird ebenfall in der Tabelle \ref{tab:regressiondouble} und \ref{tab:regressiondoublelinks} ersichtlich.
Jedoch weist der zweite Stab in diesen Tabellen ein hohes Elastizitätsmodul auf, wodurch mit Hilfe der
beidseitigen Einspannung nicht auf den Elastizitätsmodul der ersten beiden Stäbe geschlossen werden kann.
Bei dem zweiten Stab lässt sich aufgrund des Elastizitätsmodul und der Farbe lässt sich das Material Nickel vermuten.
Der dritte Stab lässt sich aufgrund des Gewichts, der Farbe und des Elastizitätsmoduls als Aluminium identifizieren.
Die Farbe und das Elastizitätsmodul des vierten Stabs deuten auf Messing hin.
In der Tabelle \ref{tab:einslit} sind die Abweichungen der Elastizitätsmodulen der einseitigen Spannung von den Literaturwerten\cite{lit} der vermuteten Materialien aufgetragen.
\begin{table}
    \centering
    \caption{Vergleich der berechneten Elastizitätsmodule mit den Literaturwerten der vermuteten Materialen bei einseitiger Einspannung}
    \label{tab:einslit}
    \begin{tabular} {S[table-format=3] S[table-format=3.2]  S[table-format=3.0] S[table-format=2.2]}
        \toprule
        & {$E$} & {$E_\text{Lit}$} & {$\eta \mathbin{/} \si{\percent}$} \\
    \midrule
    {Stab 1} & 759.77 & {-} & {-}   \\
    {Stab 2} & 228.52 & 205 & 11.47 \\
    {Stab 3} & 90.4   & 70  & 29.14 \\
    {Stab 4} & 152.09 & 123 & 23.65 \\
    \bottomrule
    \end{tabular}
\end{table}
In der Tabelle \ref{tab:zweilit} sind die Abweichungen von den beidseitigen zu den Literaturwerten\cite{lit} von den vermuteten Materialen aufgeführt
\begin{table}
    \centering
    \caption{Vergleich der berechneten Elastizitätsmodule mit den Literaturwerten der vermuteten Materialen bei beidseitiger Einspannung}
    \label{tab:zweilit}
    \begin{tabular} {S[table-format=3] S[table-format=3.2] S[table-format=3.2] S[table-format=3.0]  S[table-format=3.2] S[table-format=3.2]}
        \toprule
        & {$E_\text{R}$} & {$E_\text{L}$} & {$E_\text{Lit}$} & {$\eta_\text{R} \mathbin{/} \si{\percent}$} & {$\eta_\text{L} \mathbin{/} \si{\percent}$}\\
    \midrule
    {Stab 1} & 485.49 & 421.36 & {-} & {-}    & {-}       \\
    {Stab 2} & 436.15 & 432.82 & 205 & 112.76 & 111.13    \\
    {Stab 3} & 110.04 & 81.78  & 70 & 57.2   & 16.83     \\
    {Stab 4} & 222.31 & 216.84 & 123 & 80.73  & 76.29     \\
    \bottomrule
    \end{tabular}
\end{table}
Bei dem Vergleich von den beiden Methoden wird auffällig, dass die beidseitige Einspannung bis auf den ersten Stabs größere Elastizitätsmodule aufweist, so dass
sich die Vermutung aufstellen lässt, dass dort ein systematischer Zusammenhang vorliegt.
Im Allgemeinen lässt sich anmerken, dass die Messuhren sehr empdfindlich waren und der Messvorgang bei den runden Stäben dadurch erschwert wurde, dass der Taster
der Messuhren von den runden Stäben abrutschte, so dass dort Abweichungen enstanden sein könnten. 