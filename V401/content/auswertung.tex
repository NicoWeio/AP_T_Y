\section{Auswertung}
\label{sec:Auswertung}
Jegliche Fehlerrechnung wurde mit der Python-Bibliothek uncertainties \cite{uncertainties} absolviert. Trotz dessen sind die Formeln für
die Unsicherheiten in den jeweiligen Abschnitten angegeben. Allgemeine Rechnungen wurden mit der Python-Bibliothek numpy \cite{numpy} automatisiert.
\subsection{Bestimmung der Wellenlänge mittels Verschiebung eines Spiegels}
Die Wellenlänge der Quelle lässt sich mittels der Gleichung \eqref{eqn:wel} ermitteln.
Um die tatsächliche Weglänge $\symup{\Delta} d_\text{Ü}$ zu berechnen, muss die an der Mikrometerschraube gemessene Weglänge $\symup{\Delta}d$ 
zuächst durch den Übersetzungsfaktor $\text{Ü} = \num{5.046}$ geteilt werden.
In der Tabelle \ref{tab:wavelengths} sind die gemessenen Weglängen $\symup{\Delta} d$ bzw. $\symup{\Delta} d_\text{Ü}$ und Impulse $z$ mit den daraus errechneten Wellenlängen
$\lambda$ sowie dessen Unsicherheit $\symup{\Delta}\lambda$ aufgetragen.
\begin{table}
    \centering
    \caption{Gemessene Weglängen und Impulse mit den berechneten Wellenlängen}
    \label{tab:wavelengths}
    \begin{tabular}{S[table-format = 1.2] S[table-format = 1.2] S[table-format = 4] S[table-format = 4.2] S[table-format = 3.2]}
        \toprule
        {$\symup{\Delta} d \mathbin{/} \si{\milli\metre}$} & {$\symup{\Delta} d_\text{Ü} \mathbin{/} \si{\milli\metre}$} & {$z$} 
        & {$\lambda \mathbin{/} \si{\nano\metre}$} & {$\symup{\Delta}\lambda \mathbin{/} \si{\nano\metre}$}   \\
        \midrule
        6.39 & 1.27 & 3779 & 670.20 & 70.22\\
        6.39 & 1.27 & 3927 & 644.95 & 70.22\\
        5.13 & 1.02 & 4721 & 430.69 & 70.22\\
        5    & 0.99 & 3019 & 656.43 & 70.22\\
        5.45 & 1.08 & 3723 & 580.21 & 70.22\\
        5.45 & 1.08 & 3267 & 661.20 & 70.22\\
        5    & 0.99 & 2943 & 673.38 & 70.22\\
        5    & 0.99 & 3023 & 655.56 & 70.22\\
        5    & 0.99 & 2997 & 661.25 & 70.22\\
        5    & 0.99 & 3086 & 642.18 & 70.22\\    
        \bottomrule
    \end{tabular}
\end{table}
Aus den errechneten Wellenlängen ergibt sich ein Mittelwert von 
\begin{equation*}
    \bar{\lambda} = \SI{658.14(386)}{\nano\metre} \; \text{.}
\end{equation*}
In der Berechnung des Mittelwerts wurden der dritte und fünfte Wert ausgelassen, da diese eine zu starke Abweichung mit sich bringen.
\subsection{Berechnung des Brechungsindex von Luft}
Für jeden Messdurchgang wurde eine Druckdifferenz von $\symup{\Delta}p = \SI{0.7999}{\bar}$ verwendet.
Für die Berechnung wurden die Normalbedingungen $T_0 = \SI{273.15}{\kelvin}$ und $p_0 = \SI{1.0132}{\bar{}}$ angenommen.
Die Messungen wurden bei Raumtemperatur $T = \SI{293.15}{\kelvin}$ durchgeführt.
In der Tabelle \ref{tab:fractionindex} sind die gemessenen Impulse $z$ für jeden Messdurchgang und die daraus nach
Gleichung \eqref{eqn:brechen} errechneten Brechungsindizes $n$ aufgetragen.
\begin{table}
    \centering
    \caption{Gemessene Impulse je Druckänderungsvorgang}
    \label{tab:fractionindex}
    \begin{tabular}{S[table-format = 2.0] S[table-format = 1.6] S[table-format = 1.6]}
        \toprule
        {$z$} & {$n$} & {$\symup{\Delta} n $}\\
        \midrule
        42 & 1.000362 & 0.000186 \\
        67 & 1.000578 & 0.000186 \\
        18 & 1.000155 & 0.000186 \\
        52 & 1.000448 & 0.000186 \\
        17 & 1.000146 & 0.000186 \\
        66 & 1.000569 & 0.000186 \\
        16 & 1.000138 & 0.000186 \\
        62 & 1.000535 & 0.000186 \\       
        \bottomrule
    \end{tabular}
\end{table}
Daraus lässt sich der Mittelwert des Brechungsindex von Luft zu 
\begin{equation*}
    \bar{n} = \num{1.000367(69)}
\end{equation*}
errechnen. \\
Die Standardabweichung eines Messwertes $x$ wird bei $N$ Messwerten gemäß 
\begin{equation}
    \symup{\Delta} x = \sqrt{\frac{1}{(N-1)} \sum_{i=1}^N (x_i - \bar{x})^2}
\end{equation}
berechnet. Zur Berechnung der Standardabweichung des Mittelwerts $\bar{x}$ wird die Gleichung
\begin{equation}
    \symup{\Delta} \bar{x} = \sqrt{\frac{1}{N(N-1)} \sum_{i=1}^N (x_i - \bar{x})^2}
\end{equation} 
benötigt.